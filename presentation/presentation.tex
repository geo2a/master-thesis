\input{preamble.tex}

\usepackage[cache=false]{minted}
\usepackage{pgf}
\usepackage{tikz}

\title[Структурирование вычислений с эффектами]{Структурирование вычислений с эффектами}

\subtitle{}

% \author[Георгий Лукьянов]{%
% Георгий Лукьянов\texorpdfstring{\\}{ }
% georgiy.lukjanov@gmail.com}

% \author[]{%
% Георгий Лукьянов\texorpdfstring{\\}{ }
% \textit{georgiy.lukjanov@gmail.com}\texorpdfstring{\\}{ }
% Артём Пеленицын \texorpdfstring{\\}{ }
% \textit{apel@sfedu.ru}
% }


\author[Г.\,А.\,Лукьянов]
{%
  % \texorpdfstring{
  %   \begin{columns}
  %     % \column{.45\linewidth}
  %     % \centering
  %     % John Doe\          \href{mailto:john@example.com}{john@example.com}
  %     % \column{.45\linewidth}
  %     % \centering
  %     % Jane Doe\          \href{mailto:jane.doe@example.com}{jane.doe@example.com}
  %     \column{.45\linewidth}
  %     \centering
  %     Георгий Лукьянов\\ \scriptsize{\textit{georgiy.lukjanov@gmail.com}}
  %     \column{.45\linewidth}
  %     \centering
  %     Артём Пеленицын\\ \scriptsize{\textit{apel@sfedu.ru}}
  %   \end{columns}
  % }
  {Г.\,А.\,Лукьянов \\
  \scriptsize{\emph{Научный руководитель}: к.т.н., с.н.с, доц.каф. ИВЭ А.\,Н.\,Литвиненко}}
}

\date{21 Июня 2017}%

\institute[]{%
Южный Федеральный Университет \texorpdfstring{\\}{ }
Институт математики, механики и компьютерных наук имени~И.\,И.\,Воровича\texorpdfstring{\\}{ }
}

\begin{document}

\begin{frame}
\titlepage
\end{frame}

%%%%%%%%%%%%%%%%%%%%%%%%%%%%%%%%%%%%%%%%%%%%%%%%%%%%%%%%%%%%%%%%%%%%%%%%%%%%%%%%
%%%%%%%%%%%%%%%%%  Intro  %%%%%%%%%%%%%%%%%%%%%%%%%%%%%%%%%%%%%%%%%%%%%%%%%%%%%%
%%%%%%%%%%%%%%%%%%%%%%%%%%%%%%%%%%%%%%%%%%%%%%%%%%%%%%%%%%%%%%%%%%%%%%%%%%%%%%%%

\begin{frame}[fragile]{Постановка задачи}
\begin{block}{Цель}
\emph{Сравнительный анализ} трёх реализаций \emph{систем контроля вычислительных эффектов}: преобразователи монад в Haskell, расширяемые эффекты в Haskell, алгебраические эффекты во Frank.
\end{block}
\begin{block}{Задачи}
\begin{enumerate}
\item \emph{Реализация парсер-комбинаторных библиотек} с использованием трёх
      систем контроля эффектов.
\item \emph{Сравнение производительности} и~\emph{определение качественных отличий} полученных
      реализаций парсер-комбинаторных библиотек.
\item \emph{Дизайн и реализация клиент-серверного приложения} для учёта студенческой
      активности с использованием различных подходов к контролю над эффектами.
\end{enumerate}
\end{block}
\end{frame}

\section{Контроль вычислительных эффектов}

\begin{frame}[fragile]{Вычислительные эффекты}
  \begin{block}{Функция с побочным эффектом в языке без системы эффектов}
  \begin{minted}{csharp}
int plus(int x, int y) {
  print("Неструктурированный побочный эффект.");
  return x + y;
}
  \end{minted}
  \end{block}
  \begin{block}{Чистая функция в Haskell}
  \begin{minted}{haskell}
plus :: Int -> Int -> Int
plut x y = x + y
  \end{minted}
  \end{block}
  \begin{block}{Функция с побочным эффектом в Haskell}
  \begin{minted}{haskell}
plusIO :: Int -> Int -> IO Int
plutIO x y = do
  print "Структурированный побочный эффект."
  return (x + y)
  \end{minted}
  \end{block}
\end{frame}

\begin{frame}[fragile]{Подходы к структурированию эффектов и их реализации}

\begin{columns}
\begin{column}{0.49\textwidth}
  \begin{block}{Монадический подход}
    \begin{itemize}
      \item Монады и преобразователи монад в Haskell
    \end{itemize}
    \vspace{0.01em}
  \end{block}
\end{column}
\begin{column}{0.49\textwidth}
  \begin{block}{Алгебраические эффекты}
    \begin{itemize}
      \item Расширяемые эффекты в Haskell
      \item Язык программирования Frank
    \end{itemize}
  \end{block}
\end{column}
\end{columns}
\end{frame}

\section{Комбинаторы парсеров: модельная задача структурирования эффектов}

\begin{frame}[fragile]{Структура эффектов парсера}
\begin{itemize}
  \item Изменяемое состояние входного потока
  \item Потенциальная невозможность или неоднозначность разбора
\end{itemize}
% \begin{block}{}
% \begin{minted}{haskell}
% newtype Parser t a =
%   Parser (StateT (ParserState t) (Either (Error t)) a)
%     deriving ( Functor, Applicative, Monad
%              , MonadState (ParserState t)
%              , MonadError (Error t)
%              )
% \end{minted}
% \end{block}
\end{frame}

\begin{frame}[fragile]{Парсер как монадический стек}
\begin{block}{}
\begin{minted}{haskell}
newtype Parser a =
  Parser (StateT String (Either Error a)

parse :: Parser a -> String -> Either Error (a, String)
parse (Parser p) s = runStateT p s
\end{minted}
\end{block}
\end{frame}

\begin{frame}[fragile]{Парсер на основе расширяемых эффектов}
\begin{block}{}
\begin{minted}{haskell}
type Parsable r = (Member Fail r, Member (State String) r)

type Parser r a = Parsable r => Eff r a

parse :: Eff (Fail :> State String :> Void) a ->
         String -> (String, Maybe a)
parse p inp = run . runState inp . runFail $ p
\end{minted}
\end{block}
\end{frame}

\begin{frame}[fragile]{Frank: парсер как комбинация эффектов}
\begin{block}{}
\begin{minted}{haskell}
parse : {[Error, State String] X} ->
        String -> Maybe X
parse p str = catch (state str p!)
\end{minted}
\end{block}
\end{frame}

\begin{frame}[fragile]{Frank: парсер как монолитный эффект}
\begin{block}{Возможный интерфейс эффекта~\mintinline{haskell}{Parser}}
\begin{minted}{haskell}
interface Parser =
    fail : forall Y . Y
  | sat : {Char -> Bool} -> Char
  | choose : forall Y . {[Parser] Y} -> {[Parser] Y} -> Y
  | many : forall Y . {[Parser] Y} -> List Y
\end{minted}
\end{block}
\end{frame}

\section{Контроль эффектов для разработки реальных систем}

\begin{frame}[fragile]{Архитектура ``Student's Big Brother''}
\begin{columns}
\begin{column}{0.1\textwidth}

\end{column}
\begin{column}{0.9\textwidth}
\usetikzlibrary{shapes, arrows, calc, positioning}

\tikzset{
    module/.style={
           rectangle,
           rounded corners,
           draw=black, very thick,
           minimum height=2em,
           inner sep=2pt,
           text centered,
           },
}


\begin{tikzpicture}[->,>=stealth']

 \node[ellipse, draw=black, very thick] (TEACHER)
 {%
  \textbf{Teacher}
 };

 \node[module,
  yshift=-0.5cm,
  text width=2cm,
  below of=TEACHER] (UI)
 {%
  \textbf{Web UI}
 };

 \node[module,
  text width=2cm,
  yshift=-0.5cm,
  below of=UI] (SERVER)
 {%
  \textbf{Server}\\
 };

 \node[module,
  below of=SERVER,
  yshift=-0.5cm,
  xshift=-3cm,
  anchor=center,
  text width=2cm] (DAEMON_1)
 {%
  \textbf{Daemon}
 };

 \node[module,
  right of=DAEMON_1,
  xshift=2cm,
  text width=2cm] (DAEMON_2)
 {%
  \textbf{Daemon}
 };

 \node[module,
  right of=DAEMON_2,
  xshift=1cm,
  draw=none,
  text width=1cm] (HIDDEN_DAEMONS)
 {%
  \textbf{...}
 };

  \node[module,
  right of=HIDDEN_DAEMONS,
  xshift=1cm,
  text width=2cm] (DAEMON_3)
 {%
  \textbf{Daemon}
 };

 \node[module,
  right of=SERVER,
  xshift=3cm,
  text width=2cm] (DB)
 {%
  \textbf{DB}
 };

 \node[ellipse, draw=black, very thick,
       below of=DAEMON_1,
       yshift=-0.5cm, ] (STUDENT_1)
 {%
  \textbf{Student}
 };

  \node[ellipse, draw=black, very thick,
       below of=DAEMON_2,
       yshift=-0.5cm, ] (STUDENT_2)
 {%
  \textbf{Student}
 };

  \node[ellipse,
       below of=HIDDEN_DAEMONS,
       yshift=-0.5cm, ] (HIDDNE_STUDENT)
 {%
  \textbf{...}
 };

  \node[ellipse, draw=black, very thick,
       below of=DAEMON_3,
       yshift=-0.5cm, ] (STUDENT_3)
 {%
  \textbf{Student}
 };

 \path (DAEMON_1) edge (SERVER);
 \path (DAEMON_2) edge (SERVER);
 \path (DAEMON_3) edge (SERVER);

 \path (SERVER) edge (DB);
 \path (DB) edge (SERVER);

 \path (SERVER) edge (UI);
 \path (UI) edge (SERVER);

 \path (TEACHER) edge[dashed] (UI);

 \path (STUDENT_1) edge[dashed] (DAEMON_1);
 \path (STUDENT_2) edge[dashed] (DAEMON_2);
 \path (STUDENT_3) edge[dashed] (DAEMON_3);

\end{tikzpicture}
\end{column}
\end{columns}
\end{frame}

\begin{frame}[fragile]{Особенности реализации \\ ``Student's Big Brother''}
\begin{itemize}
\item Спецификация HTTP API на уровне типов
\item Единые типы в серверном и клиентском коде
\item Генерация кода с использованием механизмов обобщённого программирования
\item Управление побочными эффектами сервера с помощью преобразователей монад
\item Применение расширяемых эффектов в реализации агента сбора данных (daemon)
\end{itemize}
\end{frame}

\begin{frame}[fragile]{Результаты}
\begin{itemize}
\item Разработаны~\emph{библиотеки комбинаторов парсеров} на
      основе~\textbf{преобразователей монад} и~\textbf{расширяемых эффектов}.
\item \textbf{Выделены отличия} преобразователей монад и расширяемых эффектов.
\item Произведено~\textbf{сравнение производительности} разработанных библиотек с Pandoc.
\item Разработана~\textbf{серия прототипов библиотек парсеров} для Frank и выявлены
      \emph{ограничения выразительности}.
\item Разработана и апробирована~\textbf{система контроля производительности студентов}
      на лабораторных работах по программированию. В реализации использованы методы
      контроля побочных эффектов.
\end{itemize}
\end{frame}

\end{document}