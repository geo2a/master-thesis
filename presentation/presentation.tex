%% В этот файл не предполагается вносить изменения

% В этом файле следует указать информацию о себе
% и выполняемой работе.

\documentclass [fontsize=14pt, paper=a4, pagesize, DIV=calc]%
{scrreprt}
% ВНИМАНИЕ! Для использования глав поменять
% scrartcl на scrreprt

% Здесь ничего не менять
\usepackage [T2A] {fontenc}   % Кириллица в PDF файле
\usepackage [utf8] {inputenc} % Кодировка текста: utf-8
\usepackage [russian] {babel} % Переносы, лигатуры

%%%%%%%%%%%%%%%%%%%%%%%%%%%%%%%%%%%%%%%%%%%%%%%%%%%%%%%%%%%%%%%%%%%%%%%%
% Создание макроса управления элементами, специфичными
% для вида работы (курс., бак., маг.)
% Здесь ничего не менять:
\usepackage{ifthen}
\newcounter{worktype}
\newcommand{\typeOfWork}[1]
{
	\setcounter{worktype}{#1}
}

% ВНИМАНИЕ!
% Укажите тип работы: 0 - курсовая, 1 - бак., 2 - маг.,
% 3 - бакалаврская с главами.
\typeOfWork{2}
% Считается, что курсовая и бак. бьются на разделы (section) и
% подразделы (subsection), а маг. — на главы (chapter), разделы и
%  подразделы. Если хочется,
% чтобы бак. была с главами (например, если она большая),
% надо выбрать опцию 3.

% Если при выборе 2 или 3 вы забудете поменять класс
% документа на scrreprt (см. выше, в самом начале),
% то получите ошибку:
% ./aux/appearance.tex:52: Package scrbase Error: unknown option ` chapterprefix=

%%%%%%%%%%%%%%%%%%%%%%%%%%%%%%%%%%%%%%%%%%%%%%%%%%%%%%%%%%%%%%%%%%%%%%%%
% Информация об авторе и работе для титульной страницы

\usepackage {titling}

% Имя автора в именительном падеже (для маг.)
\newcommand {\me}{%
G.\,A.~Lukyanov%
}

% Имя автора в родительном падеже (для курсовой и бак.)
\newcommand {\byme}{%
И.\,И.~Иванова%
}

% Любимый научный руководитель
\newcommand{\supervisor}%
{учёная степень, учёное звание /  должность И. О. Фамилия}

% идентифицируем пол (только для курсовой и бак.)
\newcommand{\bystudent}{
Студента %Студентки % Для курсовой: с большой буквы
}

% Год публикации
\date{2017}

% Название работы
\title{Constructing effectful computations}

% Кафедра
%
\newboolean{needchair}
\setboolean{needchair}{false} % на ФИИТ не пишется (false), на ПМИ есть (true)

\newcommand {\thechair} {%
Кафедра компьютерного и аналогового моделирования светлого будущего%
}

\newcommand {\direction} {%
Направление подготовки\\
Фундаментальная информатика и информационные технологии%
}% Прикладная математика и информатика

%%%%%%%%%%%%%%%%%%%%%%%%%%%%%%%%%%%%%%%%%%%%%%%%%%%%%%%%%%%%%%%%%%%%%%%%
% Другие настраиваемые элементы текста

% Листинги с исходным кодом программ: укажите язык программирования
\usepackage{listings}
\lstset{
    language=Haskell,%  Язык указать здесь
    basicstyle=\small\ttfamily,
    breaklines=true,%
    showstringspaces=false%
    inputencoding=utf8x%
}
% полный список языков, поддерживаемых данным пакетом, есть,
% например, здесь (стр. 13):
% ftp://ftp.tex.ac.uk/tex-archive/macros/latex/contrib/listings/listings.pdf

% Нумерация списков: можно при необходимести
% изменять вид нумерации (например, добавлять правую скобку).
% По умолчанию буду списки вида:
% 1.
% 2.
% Изменять вид нумерации можно в начале нумерации:
% \begin{enumerate}[1)] (В квадратных скобках указан желаемый вид)
\usepackage[shortlabels]{enumitem}
                    \setlist[enumerate, 1]{1.}

% Гиперссылки: настройте внешний вид ссылок
\usepackage%
[pdftex,unicode,pdfborder={0 0 0},draft=false,%backref=page,
    hidelinks, % убрать, если хочется видеть ссылки: это
               % удобно в PDF файле, но не должно появиться на печати
    bookmarks=true,bookmarksnumbered=false,bookmarksopen=false]%
{hyperref}


\usepackage {amsmath}      % Больше математики
\usepackage {amssymb}
\usepackage {textcase}     % Преобразование к верхнему регистру
\usepackage {indentfirst}  % Красная строка первого абзаца в разделе
\usepackage [super]{nth}

\usepackage {fancyvrb}     % Листинги: определяем своё окружение Verb
\DefineVerbatimEnvironment% с уменьшенным шрифтом
	{Verb}{Verbatim}
	{fontsize=\small}

% Вставка рисунков
\usepackage {graphicx}

% Общее оформление
% ----------------------------------------------------------------
% Настройка внешнего вида

%%% Шрифты

% если закомментировать всё — консервативная гарнитура Computer Modern
\usepackage{paratype} % профессиональные свободные шрифты
%\usepackage {droid}  % неплохие свободные шрифты от Google
%\usepackage{mathptmx}
%\usepackage {mmasym}
%\usepackage {psfonts}
%\usepackage{lmodern}
%var1: lh additions for bold concrete fonts
%\usepackage{lh-t2axccr}
%var2: the package below could be covered with fd-files
%\usepackage{lh-t2accr}
%\usepackage {pscyr}

% Геометрия текста

\usepackage{setspace}       % Межстрочный интервал
\onehalfspacing

\newlength\MyIndent
\setlength\MyIndent{1.25cm}
\setlength{\parindent}{\MyIndent} % Абзацный отступ
\frenchspacing            % Отключение лишних отступов после точек
\KOMAoptions{%
    DIV=calc,         % Пересчёт геометрии
    numbers=endperiod % точки после номеров разделов
}

                            % Консервативный вариант:
%\usepackage                % ручное задание геометрии
%[%                         % (не рекомендуется в проф. типографии)
%  margin = 2.5cm,
  %includefoot,
  %footskip = 1cm
%] %
%  {geometry}

%%% Заголовки

\ifthenelse{\equal{\theworktype}{2}}{%
\KOMAoptions{%
    numbers=endperiod,% точки после номеров разделов
    headings=normal,   % размеры заголовков поменьше стандартных
    chapterprefix=true,% Печатать слово Глава в магистерской
    appendixprefix=true% Печатать слово Приложение
}
}

% шрифт для оформления глав и названия содержания
\newcommand{\SuperFont}{\Large\sffamily\bfseries}

% Заголовок главы
\ifthenelse{\value{worktype} > 1}{%
\renewcommand{\SuperFont}{\Large\normalfont\sffamily}
\newcommand{\CentSuperFont}{\centering\SuperFont}
\usepackage{fncychap}
\ChNameVar{\SuperFont}
\ChNumVar{\CentSuperFont}
\ChTitleVar{\CentSuperFont}
\ChNameUpperCase
\ChTitleUpperCase
}

% Заголовок (под)раздела с абзацного отступа
\addtokomafont{sectioning}{\hspace{\MyIndent}}

\renewcommand*{\captionformat}{~---~}
\renewcommand*{\figureformat}{Listing~\thefigure}

% Плавающие листинги
\usepackage{float}
\floatstyle{ruled}
\floatname{ListingEnv}{Листинг}
\newfloat{ListingEnv}{htbp}{lol}[section]

% точка после номера листинга
\makeatletter
\renewcommand\floatc@ruled[2]{{\@fs@cfont #1.} #2\par}
\makeatother


%%% Оглавление
\usepackage{tocloft}

% шрифт и положение заголовка
\ifthenelse{\value{worktype} > 1}{%
\renewcommand{\cfttoctitlefont}{\hfil\SuperFont\MakeUppercase}
}{
\renewcommand{\cfttoctitlefont}{\hfil\SuperFont}
}

% слово Глава
\usepackage{calc}
\ifthenelse{\value{worktype} > 1}{%
\renewcommand{\cftchappresnum}{Chapter }
\addtolength{\cftchapnumwidth}{\widthof{Chapter }}
}

\newcommand{\setupname}[1]{%
  \addtocontents{toc}{%
    \unexpanded{\unexpanded{%
      \renewcommand{\cftchappresnum}{#1 }%
      \setlength\cftchapnumwidth{\widthof{\bfseries #1 }}%
      \addtolength\cftchapnumwidth{\fixedchapnumwidth}%
    }}%
  }%
}
\AtBeginDocument{\edef\fixedchapnumwidth{\the\cftchapnumwidth}}

% Очищаем оформление названий старших элементов в оглавлении
\ifthenelse{\value{worktype} > 1}{%
\renewcommand{\cftchapfont}{}
\renewcommand{\cftchappagefont}{}
}{
\renewcommand{\cftsecfont}{}
\renewcommand{\cftsecpagefont}{}
}

% Точки после верхних элементов оглавления
\renewcommand{\cftsecdotsep}{\cftdotsep}
%\newcommand{\cftchapdotsep}{\cftdotsep}

\ifthenelse{\value{worktype} > 1}{%
    \renewcommand{\cftchapaftersnum}{.}
}{}
\renewcommand{\cftsecaftersnum}{.}
\renewcommand{\cftsubsecaftersnum}{.}
\renewcommand{\cftsubsubsecaftersnum}{.}

%%% Списки (enumitem)

\usepackage {enumitem}      % Списки с настройкой отступов
\setlist %
{ %
  leftmargin = \parindent, itemsep=.5ex, topsep=.4ex
} %

% По ГОСТу нумерация должны быть буквами: а, б...
%\makeatletter
%    \AddEnumerateCounter{\asbuk}{\@asbuk}{м)}
%\makeatother
%\renewcommand{\labelenumi}{\asbuk{enumi})}
%\renewcommand{\labelenumii}{\arabic{enumii})}

%%% Таблицы: выбрать более подходящие

\usepackage{booktabs} % считаются наиболее профессионально выполненными
%\usepackage{ltablex}
%\newcolumntype {L} {>{---}l}

%%% Библиография

\usepackage{csquotes}        % Оформление списка литературы
\usepackage[
  backend=biber,
  hyperref=auto,
  sorting=none, % сортировка в порядке встречаемости ссылок
  language=auto,
  citestyle=gost-numeric,
  bibstyle=gost-numeric
]{biblatex}
\addbibresource{biblio.bib} % Файл с лит.источниками

% Настройка величины отступа в списке
\ifthenelse{\value{worktype} < 2}{%
\defbibenvironment{bibliography}
  {\list
     {\printtext[labelnumberwidth]{%
    \printfield{prefixnumber}%
    \printfield{labelnumber}}}
     {\setlength{\labelwidth}{\labelnumberwidth}%
      \setlength{\leftmargin}{\labelwidth}%
      \setlength{\labelsep}{\dimexpr\MyIndent-\labelwidth\relax}% <----- default is \biblabelsep
      \addtolength{\leftmargin}{\labelsep}%
      \setlength{\itemsep}{\bibitemsep}%
      \setlength{\parsep}{\bibparsep}}%
      \renewcommand*{\makelabel}[1]{\hss##1}}
  {\endlist}
  {\item}
}{}

% ----------------------------------------------------------------
% Настройка переносов и разрывов страниц

\binoppenalty = 10000      % Запрет переносов строк в формулах
\relpenalty = 10000        %

\sloppy                    % Не выходить за границы бокса
%\tolerance = 400          % или более точно
\clubpenalty = 10000       % Запрет разрывов страниц после первой
\widowpenalty = 10000      % и перед предпоследней строкой абзаца

% ----------------------------


% Стили для окружений типа Определение, Теорема...
% Оформление теорем (ntheorem)

\usepackage [thmmarks, amsmath] {ntheorem}
\theorempreskipamount 0.6cm

\theoremstyle {plain} %
\theoremheaderfont {\normalfont \bfseries} %
\theorembodyfont {\slshape} %
\theoremsymbol {\ensuremath {_\Box}} %
\theoremseparator {:} %
\newtheorem {mystatement} {Утверждение} [section] %
\newtheorem {mylemma} {Лемма} [section] %
\newtheorem {mycorollary} {Следствие} [section] %

\theoremstyle {nonumberplain} %
\theoremseparator {.} %
\theoremsymbol {\ensuremath {_\diamondsuit}} %
\newtheorem {mydefinition} {Определение} %

\theoremstyle {plain} %
\theoremheaderfont {\normalfont \bfseries} 
\theorembodyfont {\normalfont} 
%\theoremsymbol {\ensuremath {_\Box}} %
\theoremseparator {.} %
\newtheorem {mytask} {Задача} [section]%
\renewcommand{\themytask}{\arabic{mytask}}

\theoremheaderfont {\scshape} %
\theorembodyfont {\upshape} %
\theoremstyle {nonumberplain} %
\theoremseparator {} %
\theoremsymbol {\rule {1ex} {1ex}} %
\newtheorem {myproof} {Доказательство} %

\theorembodyfont {\upshape} %
%\theoremindent 0.5cm
\theoremstyle {nonumberbreak} \theoremseparator {\\} %
\theoremsymbol {\ensuremath {\ast}} %
\newtheorem {myexample} {Пример} %
\newtheorem {myexamples} {Примеры} %

\theoremheaderfont {\itshape} %
\theorembodyfont {\upshape} %
\theoremstyle {nonumberplain} %
\theoremseparator {:} %
\theoremsymbol {\ensuremath {_\triangle}} %
\newtheorem {myremark} {Замечание} %
\theoremstyle {nonumberbreak} %
\newtheorem {myremarks} {Замечания} %


% Титульный лист
% Макросы настройки титульной страницы
% В этот файл не предполагается вносить изменения

%\usepackage {showframe}

% Вертикальные отступы на титульной странице
\newcommand{\vgap}{\vspace{16pt}}

% Помещение города и даты в нижний колонтитул
\usepackage{scrlayer}
\DeclareNewLayer[
  foot,
  foreground,
  contents={%
    \raisebox{\dp\strutbox}[\layerheight][0pt]{%
      \parbox[b]{\layerwidth}{\centering Ростов-на-Дону\\ \thedate%
       \\\mbox{}
       }}%
  }
]{titlepage.foot.fg}
\DeclareNewPageStyleByLayers{titlepage}{titlepage.foot.fg}


\AtBeginDocument %
{ %
  %
  \begin{titlepage}
  %
    \thispagestyle{titlepage}

    {\centering
    %
    \MakeTextUppercase {МИНИСТЕРСТВО ОБРАЗОВАНИЯ И НАУКИ РФ}

    \vgap

    Федеральное государственное автономное образовательное\\
    учреждение высшего образования\\
    \MakeTextUppercase {Южный федеральный университет}

    \vgap

	Институт математики, механики и компьютерных наук
    имени~И.\,И.\,Воровича

    \vgap

    \direction

    \ifthenelse{\boolean{needchair}}{
    \vgap

    \thechair}{}

    \vspace* {\fill}

    \ifthenelse{\value{worktype} = 2}{%
    \me

    \vgap}{}

    {\usefont{T2A}{PTSansCaption-TLF}{m}{n}
    \MakeTextUppercase{\thetitle}}

    \ifthenelse{\value{worktype} = 2}{%
     \vgap

    Магистерская диссертация}{}
    \ifthenelse{\value{worktype} = 0}{
     \vgap

    Курсовая работа
    }{}%
    \ifthenelse{\value{worktype} = 1 \OR \value{worktype} = 3}{
     \vgap

    Выпускная квалификационная работа\\
    на степень бакалавра
    }{}%

    \vspace {\fill}

    \begin{flushright}
    \ifthenelse{\value{worktype} = 0 \OR
                \value{worktype} = 1 \OR
                \value{worktype} = 3}{
      \bystudent \ifthenelse{\value{worktype} = 0}{3}{4}\ курса\\
      \byme
    }{}

    \vgap

    Научный руководитель:\\
    к.т.н., с.н.с., доц. каф. ИВЭ
    А.~Н.~Литвиненко\\
    \ifthenelse{\value{worktype} = 2}{%
    Рецензент:\\
    к.ф.-м.н., доц., доц. каф. АДМ С.~С.~Михалкович}{}
	\end{flushright}
    \ifthenelse{\value{worktype} = 0}{
    \vspace{\fill}
            \begin{flushleft}
              \begin{tabular}{cc}
                \underline{\hspace{4cm}}&\underline{\hspace{5cm}}\\
                {\small оценка (рейтинг)} & {\small  подпись руководителя}\\
              \end{tabular}
            \end{flushleft}
    }{}
  	\vspace {\fill}
  %Ростов-на-Дону

    %\thedate

  }\end{titlepage}
  %
  %
  \tableofcontents
  %
  \clearpage
} %



% Команды для использования в тексте работы


% макросы для начала введения и заключения
\newcommand{\Ackns}{\addchap{Acknowledgements}}

\newcommand{\Intro}{\addchap{Introduction}}

\newcommand{\Goal}{\addchap{Goal statement}}

\newcommand{\Conc}{\addchap{Conclusion}}

% Правильные значки для нестрогих неравенств и пустого множества
\renewcommand {\le} {\leqslant}
\renewcommand {\ge} {\geqslant}
\renewcommand {\emptyset} {\varnothing}

% N ажурное: натуральные числа
\newcommand {\N} {\ensuremath{\mathbb N}}

% значок С++ — используйте команду \cpp
\newcommand{\cpp}{%
C\nolinebreak\hspace{-.05em}%
\raisebox{.2ex}{+}\nolinebreak\hspace{-.10em}%
\raisebox{.2ex}{+}%
}

% значок С# — используйте команду \cs
\newcommand{\cs}{%
C\nolinebreak\hspace{-.05em}%
\raisebox{.2ex}{\#}%
}

% Неразрывный дефис, который допускает перенос внутри слов,
% типа жёлто-синий: нужно писать жёлто"/синий.
\makeatletter
  \defineshorthand[english]{"/}{\babelhyphen{nobreak}}
  \addto\extrasenglish{
    \languageshorthands{english}
    \useshorthands{"}
  }
\makeatother



\endinput

% Конец файла


\usepackage[cache=false]{minted}
\usepackage{pgf}
\usepackage{tikz}

\title[Структурирование вычислений с эффектами]{Структурирование вычислений с эффектами}

\subtitle{}

% \author[Георгий Лукьянов]{%
% Георгий Лукьянов\texorpdfstring{\\}{ }
% georgiy.lukjanov@gmail.com}

% \author[]{%
% Георгий Лукьянов\texorpdfstring{\\}{ }
% \textit{georgiy.lukjanov@gmail.com}\texorpdfstring{\\}{ }
% Артём Пеленицын \texorpdfstring{\\}{ }
% \textit{apel@sfedu.ru}
% }


\author[Г.\,А.\,Лукьянов]
{%
  % \texorpdfstring{
  %   \begin{columns}
  %     % \column{.45\linewidth}
  %     % \centering
  %     % John Doe\          \href{mailto:john@example.com}{john@example.com}
  %     % \column{.45\linewidth}
  %     % \centering
  %     % Jane Doe\          \href{mailto:jane.doe@example.com}{jane.doe@example.com}
  %     \column{.45\linewidth}
  %     \centering
  %     Георгий Лукьянов\\ \scriptsize{\textit{georgiy.lukjanov@gmail.com}}
  %     \column{.45\linewidth}
  %     \centering
  %     Артём Пеленицын\\ \scriptsize{\textit{apel@sfedu.ru}}
  %   \end{columns}
  % }
  {Г.\,А.\,Лукьянов \\
  \scriptsize{\emph{Научный руководитель}: к.т.н., с.н.с, доц.каф. ИВЭ А.\,Н.\,Литвиненко}}
}

\date{21 Июня 2017}%

\institute[]{%
Южный Федеральный Университет \texorpdfstring{\\}{ }
Институт математики, механики и компьютерных наук имени~И.\,И.\,Воровича\texorpdfstring{\\}{ }
}

\begin{document}

\begin{frame}
\titlepage
\end{frame}

%%%%%%%%%%%%%%%%%%%%%%%%%%%%%%%%%%%%%%%%%%%%%%%%%%%%%%%%%%%%%%%%%%%%%%%%%%%%%%%%
%%%%%%%%%%%%%%%%%  Intro  %%%%%%%%%%%%%%%%%%%%%%%%%%%%%%%%%%%%%%%%%%%%%%%%%%%%%%
%%%%%%%%%%%%%%%%%%%%%%%%%%%%%%%%%%%%%%%%%%%%%%%%%%%%%%%%%%%%%%%%%%%%%%%%%%%%%%%%

\begin{frame}[fragile]{Постановка задачи}
\begin{block}{Цель}
\emph{Сравнительный анализ} трёх реализаций \emph{систем контроля вычислительных эффектов}: преобразователи монад в Haskell, расширяемые эффекты в Haskell, алгебраические эффекты во Frank.
\end{block}
\begin{block}{Задачи}
\begin{enumerate}
\item \emph{Реализация парсер-комбинаторных библиотек} с использованием трёх
      систем контроля эффектов.
\item \emph{Сравнение производительности} и~\emph{определение качественных отличий} полученных
      реализаций парсер-комбинаторных библиотек.
\item \emph{Дизайн и реализация клиент-серверного приложения} для учёта студенческой
      активности с использованием различных подходов к контролю над эффектами.
\end{enumerate}
\end{block}
\end{frame}

\section{Контроль вычислительных эффектов}

\begin{frame}[fragile]{Вычислительные эффекты}
  \begin{block}{Функция с побочным эффектом в языке без системы эффектов}
  \begin{minted}{csharp}
int plus(int x, int y) {
  print("Неструктурированный побочный эффект.");
  return x + y;
}
  \end{minted}
  \end{block}
  \begin{block}{Чистая функция в Haskell}
  \begin{minted}{haskell}
plus :: Int -> Int -> Int
plut x y = x + y
  \end{minted}
  \end{block}
  \begin{block}{Функция с побочным эффектом в Haskell}
  \begin{minted}{haskell}
plusIO :: Int -> Int -> IO Int
plutIO x y = do
  print "Структурированный побочный эффект."
  return (x + y)
  \end{minted}
  \end{block}
\end{frame}

\begin{frame}[fragile]{Подходы к структурированию эффектов и их реализации}

\begin{columns}
\begin{column}{0.49\textwidth}
  \begin{block}{Монадический подход}
    \begin{itemize}
      \item Монады и преобразователи монад в Haskell
    \end{itemize}
    \vspace{0.01em}
  \end{block}
\end{column}
\begin{column}{0.49\textwidth}
  \begin{block}{Алгебраические эффекты}
    \begin{itemize}
      \item Расширяемые эффекты в Haskell
      \item Язык программирования Frank
    \end{itemize}
  \end{block}
\end{column}
\end{columns}
\end{frame}

\section{Комбинаторы парсеров: модельная задача структурирования эффектов}

\begin{frame}[fragile]{Структура эффектов парсера}
\begin{itemize}
  \item Изменяемое состояние входного потока
  \item Потенциальная невозможность или неоднозначность разбора
\end{itemize}
% \begin{block}{}
% \begin{minted}{haskell}
% newtype Parser t a =
%   Parser (StateT (ParserState t) (Either (Error t)) a)
%     deriving ( Functor, Applicative, Monad
%              , MonadState (ParserState t)
%              , MonadError (Error t)
%              )
% \end{minted}
% \end{block}
\end{frame}

\begin{frame}[fragile]{Парсер как монадический стек}
\begin{block}{}
\begin{minted}{haskell}
newtype Parser a =
  Parser (StateT String (Either Error a)

parse :: Parser a -> String -> Either Error (a, String)
parse (Parser p) s = runStateT p s
\end{minted}
\end{block}
\end{frame}

\begin{frame}[fragile]{Парсер на основе расширяемых эффектов}
\begin{block}{}
\begin{minted}{haskell}
type Parsable r = (Member Fail r, Member (State String) r)

type Parser r a = Parsable r => Eff r a

parse :: Eff (Fail :> State String :> Void) a ->
         String -> (String, Maybe a)
parse p inp = run . runState inp . runFail $ p
\end{minted}
\end{block}
\end{frame}

\begin{frame}[fragile]{Frank: парсер как комбинация эффектов}
\begin{block}{}
\begin{minted}{haskell}
parse : {[Error, State String] X} ->
        String -> Maybe X
parse p str = catch (state str p!)
\end{minted}
\end{block}
\end{frame}

\begin{frame}[fragile]{Frank: парсер как монолитный эффект}
\begin{block}{Возможный интерфейс эффекта~\mintinline{haskell}{Parser}}
\begin{minted}{haskell}
interface Parser =
    fail : forall Y . Y
  | sat : {Char -> Bool} -> Char
  | choose : forall Y . {[Parser] Y} -> {[Parser] Y} -> Y
  | many : forall Y . {[Parser] Y} -> List Y
\end{minted}
\end{block}
\end{frame}

\section{Контроль эффектов для разработки реальных систем}

\begin{frame}[fragile]{Архитектура ``Student's Big Brother''}
\begin{columns}
\begin{column}{0.1\textwidth}

\end{column}
\begin{column}{0.9\textwidth}
\usetikzlibrary{shapes, arrows, calc, positioning}

\tikzset{
    module/.style={
           rectangle,
           rounded corners,
           draw=black, very thick,
           minimum height=2em,
           inner sep=2pt,
           text centered,
           },
}


\begin{tikzpicture}[->,>=stealth']

 \node[ellipse, draw=black, very thick] (TEACHER)
 {%
  \textbf{Teacher}
 };

 \node[module,
  yshift=-1cm,
  text width=2cm,
  below of=TEACHER] (UI)
 {%
  \textbf{Web UI}
 };

 \node[module,
  text width=3cm,
  yshift=-1cm,
  below of=UI] (SERVER)
 {%
  \textbf{Server}\\
 };

 \node[module,
  below of=SERVER,
  yshift=-1cm,
  xshift=-4cm,
  anchor=center,
  text width=3cm] (DAEMON_1)
 {%
  \textbf{Daemon}
 };

 \node[module,
  right of=DAEMON_1,
  xshift=3cm,
  text width=3cm] (DAEMON_2)
 {%
  \textbf{Daemon}
 };

 \node[module,
  right of=DAEMON_2,
  xshift=2cm,
  draw=none,
  text width=1cm] (HIDDEN_DAEMONS)
 {%
  \textbf{...}
 };

  \node[module,
  right of=HIDDEN_DAEMONS,
  xshift=2cm,
  text width=3cm] (DAEMON_3)
 {%
  \textbf{Daemon}
 };

 \node[module,
  right of=SERVER,
  xshift=3cm,
  text width=2cm] (DB)
 {%
  \textbf{DB}
 };

 \node[ellipse, draw=black, very thick,
       below of=DAEMON_1,
       yshift=-1cm, ] (STUDENT_1)
 {%
  \textbf{Student}
 };

  \node[ellipse, draw=black, very thick,
       below of=DAEMON_2,
       yshift=-1cm, ] (STUDENT_2)
 {%
  \textbf{Student}
 };

  \node[ellipse,
       below of=HIDDEN_DAEMONS,
       yshift=-1cm, ] (HIDDNE_STUDENT)
 {%
  \textbf{...}
 };

  \node[ellipse, draw=black, very thick,
       below of=DAEMON_3,
       yshift=-1cm, ] (STUDENT_3)
 {%
  \textbf{Student}
 };

 \path (DAEMON_1) edge (SERVER);
 \path (DAEMON_2) edge (SERVER);
 \path (DAEMON_3) edge (SERVER);

 \path (SERVER) edge (DB);
 \path (DB) edge (SERVER);

 \path (SERVER) edge (UI);
 \path (UI) edge (SERVER);

 \path (TEACHER) edge[dashed] (UI);

 \path (STUDENT_1) edge[dashed] (DAEMON_1);
 \path (STUDENT_2) edge[dashed] (DAEMON_2);
 \path (STUDENT_3) edge[dashed] (DAEMON_3);

\end{tikzpicture}
\end{column}
\end{columns}
\end{frame}

\begin{frame}[fragile]{Особенности реализации \\ ``Student's Big Brother''}
\begin{itemize}
\item Спецификация HTTP API на уровне типов
\item Единые типы в серверном и клиентском коде
\item Генерация кода с использованием механизмов обобщённого программирования
\item Управление побочными эффектами сервера с помощью преобразователей монад
\item Применение расширяемых эффектов в реализации агента сбора данных (daemon)
\end{itemize}
\end{frame}

\begin{frame}[fragile]{Результаты}
\begin{itemize}
\item Разработаны~\emph{библиотеки комбинаторов парсеров} на
      основе~\textbf{преобразователей монад} и~\textbf{расширяемых эффектов}.
\item \textbf{Выделены отличия} преобразователей монад и расширяемых эффектов.
\item Произведено~\textbf{сравнение производительности} разработанных библиотек с Pandoc.
\item Разработана~\textbf{серия прототипов библиотек парсеров} для Frank и выявлены
      \emph{ограничения выразительности}.
\item Разработана и апробирована~\textbf{система контроля производительности студентов}
      на лабораторных работах по программированию. В реализации использованы методы
      контроля побочных эффектов.
\end{itemize}
\end{frame}

\end{document}