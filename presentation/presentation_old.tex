\input{preamble.tex}

\usepackage[cache=false]{minted}

\title[]{Структурирование вычислений с эффектами}

\subtitle{}

% \author[Георгий Лукьянов]{%
% Георгий Лукьянов\texorpdfstring{\\}{ }
% georgiy.lukjanov@gmail.com}

% \author[]{%
% Георгий Лукьянов\texorpdfstring{\\}{ }
% \textit{georgiy.lukjanov@gmail.com}\texorpdfstring{\\}{ }
% Артём Пеленицын \texorpdfstring{\\}{ }
% \textit{apel@sfedu.ru}
% }


\author[Г.\,А.\,Лукьянов \& А.\,М.\,Пеленицын]
{%
  % \texorpdfstring{
  %   \begin{columns}
  %     % \column{.45\linewidth}
  %     % \centering
  %     % John Doe\          \href{mailto:john@example.com}{john@example.com}
  %     % \column{.45\linewidth}
  %     % \centering
  %     % Jane Doe\          \href{mailto:jane.doe@example.com}{jane.doe@example.com}
  %     \column{.45\linewidth}
  %     \centering
  %     Георгий Лукьянов\\ \scriptsize{\textit{georgiy.lukjanov@gmail.com}}
  %     \column{.45\linewidth}
  %     \centering
  %     Артём Пеленицын\\ \scriptsize{\textit{apel@sfedu.ru}}
  %   \end{columns}
  % }
  {Г.\,А.\,Лукьянов \& А.\,М.\,Пеленицын}
}

\date{21 Июня 2017}%

\institute[]{%
Южный Федеральный Университет \texorpdfstring{\\}{ }
Институт Математики, Механики и Компьютерных Наук имени~И.\,И.\,Воровича\texorpdfstring{\\}{ }
}

\begin{document}

\begin{frame}
\titlepage
\end{frame}

%%%%%%%%%%%%%%%%%%%%%%%%%%%%%%%%%%%%%%%%%%%%%%%%%%%%%%%%%%%%%%%%%%%%%%%%%%%%%%%%
%%%%%%%%%%%%%%%%%  Intro  %%%%%%%%%%%%%%%%%%%%%%%%%%%%%%%%%%%%%%%%%%%%%%%%%%%%%%
%%%%%%%%%%%%%%%%%%%%%%%%%%%%%%%%%%%%%%%%%%%%%%%%%%%%%%%%%%%%%%%%%%%%%%%%%%%%%%%%

\begin{frame}[fragile]{Постановка задачи}
\begin{itemize}
\item Разработать библиотеку монадических комбинаторов парсеров для
иллюстрации использования преобразователей монад в Haskell.
\item Разработать библиотеку комбинаторов парсеров на основе расширяемых эффектов~---
реализации концепции расширяемых эффектов в Haskell.
\item Выделить отличия преобразователей монад и расширяемых эффектов.
\item Проанализировать возможности экспериментального языка программирования Frank,
поддерживающего концепцию алгебраических эффектов на уровне ядра, для реализации
библиотек комбинаторов парсеров.
\item Произвести анализ производительности разработанных библиотек и сравнить с
популярными аналогами.
\item Продемонстрировать преимущества типизированного функционального языка программирования
с системой эффектов при реализации крупных систем: разработать полномасштабное
веб-приложение на Haskell.
\end{itemize}
\end{frame}

\section{Управление побочными эффектами}

% \begin{frame}{Подходы к управлению побочными эффектами (не является классификацией)}
% \begin{itemize}
%   \item Функторы
%   \item Аппликативные функторы
%   \item Монады (И трансформеры монад)
%   \item Стрелки
%   \item Алгебраические эффекты и обработчики эффектов
% \end{itemize}
% \end{frame}

\begin{frame}[fragile]{Вычислительные эффекты}
\begin{columns}
\begin{column}{0.55\textwidth}
  \begin{block}{}
  \begin{minted}{haskell}
  add : Int -> Int -> Int
  \end{minted}
  \end{block}
  \begin{block}{}
  \begin{minted}{haskell}
  addSt : State (Int, Int) Int
  \end{minted}
  \end{block}
  \begin{block}{}
  \begin{minted}{haskell}
  addIO : IO Int
  \end{minted}
  \end{block}
\end{column}
\begin{column}{0.4\textwidth}
  \begin{itemize}
    \item Контролируемые
    \begin{itemize}
      \item \texttt{State}
      \item \texttt{Reader}
      \item \texttt{Exception}
      \item \texttt{...}
    \end{itemize}
    \item Неконтролируемые
    \begin{itemize}
      \item \texttt{IO}
    \end{itemize}
  \end{itemize}
\end{column}
\end{columns}
\end{frame}

%%%%%%%%%%%%%%%%%%%%%%%%%%%%%%%%%%%%%%%%%%%%%%%%%%%%%%%%%%%%%%%%%%%%%%%%%%%%%%%%
%%%%%%%%%%%%%%%%%  Frank  %%%%%%%%%%%%%%%%%%%%%%%%%%%%%%%%%%%%%%%%%%%%%%%%%%%%%%
%%%%%%%%%%%%%%%%%%%%%%%%%%%%%%%%%%%%%%%%%%%%%%%%%%%%%%%%%%%%%%%%%%%%%%%%%%%%%%%%

\section{Язык программирования Frank}

\begin{frame}{Язык программирования Frank}
  Последняя версия представлена на POPL'17 в статье Do be do be do от Sam Lindley, Conor McBride, Craig McLaughlin
  \begin{block}{Особенности}
    \begin{itemize}
      \item \textbf{Строгая} стратегия вычислений
      \item Алгебраические типы данных
      \item Разделение \textbf{типов-значений} и \textbf{типов-вычислений}
      \item \textbf{Интерфейсы} --- сигнатуры эффектов
      \item \textbf{Операции} --- обработчики эффектов
      \item Обычные функции являются тривиальными операторами, обрабатывающими пустой множество эффектов
      \item Полиморфизм эффектов основанный на понятии \textbf{охватывающего поля эффектов} (англ. ambient ability)
    \end{itemize}
  \end{block}
\end{frame}

\subsection{Базовые конструкции}

\begin{frame}[fragile]{}
\begin{block}{Натуральные числа}
\begin{minted}{haskell}
data Nat = zero | suc Nat
\end{minted}
\end{block}
\begin{block}{Рекурсивные функции}
\begin{minted}{haskell}
eqNat : Nat -> Nat -> Bool
eqNat zero zero  = true
eqNat (suc x) (suc y) = eqNat x y
eqNat zero _     = false
eqNat _  zero    = false
\end{minted}
\end{block}
\end{frame}

\subsection{Типы-значения и типы-вычисления}

\begin{frame}[fragile]{}
\begin{block}{Вычисляющая условная операция}
\begin{minted}{haskell}
iffi : Bool -> X -> X -> X
iffi true  t f = t
iffi false t f = f
\end{minted}
\end{block}
\begin{block}{Стандартная условная операция}
\begin{minted}{haskell}
if : Bool -> {X} -> {X} -> X
if true  t f = t!
if false t f = f!
\end{minted}
\end{block}
\end{frame}

% \begin{frame}[fragile]{}
% \begin{block}{Функция \texttt{map} и отложенные вычисления}
% \begin{minted}{haskell}
% map : {X -> Y} -> List X -> List Y
% map f nil         = nil
% map f (cons x xs) = cons (f x) (map f xs)
% \end{minted}
% \end{block}
% \begin{block}{Применение \texttt{map} в отсутствие эффектов}
% \begin{minted}{haskell}
% pureMap : {List Int}
% pureMap = map {x -> x + 1} [1,2,3]
% \end{minted}
% \end{block}
% \end{frame}

\subsection{Контроль побочных эффектов}

\begin{frame}[fragile]{Интерфейсы эффектов}
\begin{block}{Эффект изменяемого состояния}
\begin{minted}{haskell}
interface State S = get : S
                  | put : S -> Unit
\end{minted}
\end{block}
\begin{block}{Эффект потенциально ошибочных вычислений}
\begin{minted}{haskell}
interface Error = fail : Zero
\end{minted}
\end{block}
\end{frame}

\begin{frame}[fragile]{Обработчики эффектов}
\begin{block}{Обработчик эффекта \texttt{State}}
\begin{minted}{haskell}
state : S -> <State S>X -> X
state _ x             = x
state s <get -> k>    = state s (k s)
state _ <put s -> k>  = state s (k unit)
\end{minted}
\end{block}
\begin{block}{Обработчик эффекта \texttt{Error}}
\begin{minted}{haskell}
catch : <Error>X -> Maybe X
catch x = just x
catch <fail -> _> = nothing
\end{minted}
\end{block}
\end{frame}



% \begin{frame}[fragile]{Управление потоком выполнения}
% \begin{block}{Энергичный условный оператор}
% \begin{minted}{haskell}
% iffy : Bool -> X -> X -> X
% iffy tt t f = t
% iffy ff t f = f
% \end{minted}
% \end{block}
% \begin{block}{``Традиционный'' условный оператор}
% \begin{minted}{haskell}
% if : Bool -> {X} -> {X} -> X
% if tt t f = t!
% if ff t f = f!
% \end{minted}
% \end{block}
% \end{frame}

% \begin{frame}[fragile]{Эффект изменяемого состояния}
% \begin{block}{Интерфейс эффекта \texttt{State}}
% \begin{minted}{haskell}
% interface State S = get : S
%                   | put : S -> Unit
% \end{minted}
% \end{block}
% \begin{block}{Обработчик эффекта \texttt{State}}
% \begin{minted}{haskell}
% state : S -> <State S>X -> X
% state _ x          = x
% state s <get -> k>  = state s (k s)
% state _ <put s -> k>  = state s (k unit)
% \end{minted}
% \end{block}
% \end{frame}

% \begin{frame}[fragile]{Исключения}
% \begin{block}{Интерфейс эффекта \texttt{Exception}}
% \begin{minted}{haskell}
% interface Exception E
%   = exception : E -> Zero
% \end{minted}
% \end{block}
% \begin{block}{Возбуждение исключения}
% \begin{minted}{haskell}
% throw : E -> [Exception E]X
% throw e = on (exception e) {}
% \end{minted}
% \end{block}
% \begin{block}{Обработка исключения}
% \begin{minted}{haskell}
% catch : <Exception E>X -> Either E X
% catch x = right x
% catch <exception e -> _> = left e
% \end{minted}
% \end{block}

% \end{frame}





%%%%%%%%%%%%%%%%%%%%%%%%%%%%%%%%%%%%%%%%%%%%%%%%%%%%%%%%%%%%%%%%%%%%%%%%%%%%%%%%
%%%%%%%%%%%%%  Building Parsers Combinators with Frank  %%%%%%%%%%%%%%%%%%%%%%%%
%%%%%%%%%%%%%%%%%%%%%%%%%%%%%%%%%%%%%%%%%%%%%%%%%%%%%%%%%%%%%%%%%%%%%%%%%%%%%%%%

\section{Построение парсеров на Frank}

\subsection{Парсер как комбинация эффектов}

\begin{frame}[fragile]{Парсер как комбинация эффектов}

\begin{block}{Обработка комбинации эффектов}
\begin{minted}{haskell}
parse : {[Error, State (List Char)] X} ->
        (List Char) -> Maybe X
parse p str = catch (state str p!)
\end{minted}
\end{block}

\pause

\begin{block}{Базовые парсеры}
\begin{minted}{haskell}
item : [Error, State (List Char)] Char
item! = on get! { nil       -> fail
                | (x :: xs) -> put xs; x}

sat : {Char -> [Error, State (List Char)] Bool} ->
      [Error, State (List Char)] Char
sat p = on item! {c -> if (p c) {c} {fail}}
\end{minted}
\end{block}
\end{frame}

\begin{frame}[fragile]{Парсер как комбинация эффектов}
\begin{block}{Парсер для заданного символа}
\begin{minted}{haskell}
char : Char -> [Error, State (List Char)] Char
char c = sat {x -> eqChar x c}
\end{minted}
\end{block}

\begin{block}{Парсер для заданной строки}
\begin{minted}{haskell}
string : (List Char) ->
         [Error, State (List Char)] (List Char)
string str = map char str
\end{minted}
\end{block}
\end{frame}

\begin{frame}[fragile]{Парсер как комбинация эффектов}
\begin{block}{Комбинатор детерминированного выбора}
\begin{minted}{haskell}
choose : {[Error, State (List Char)] X} ->
         {[Error, State (List Char)] X} ->
          [Error, State (List Char)] X
choose p1 p2 =
  on (parse p1 get!) { (right _)  -> p1!
                     | (left  _)  ->  on (parse p2 get!)
                       { (right _)   -> p2!
                       | (left err)  -> fail
                       }
                     }
\end{minted}
\end{block}
\end{frame}

\begin{frame}[fragile]{Парсер как комбинация эффектов}
\begin{block}{Комбинаторы последовательности}
\begin{minted}{haskell}
many : {[Error, State (List Char)]X} ->
        [Error, State (List Char)](List X)
many p = choose {some p} {nil}

some : {[Error, State (List Char)] X} ->
        [Error, State (List Char)](List X)
some p = p! :: many p
\end{minted}
\end{block}
\end{frame}





\subsection{Парсер как монолитный эффект}

\begin{frame}[fragile]{}
\begin{block}{Интерфейс эффекта}
\begin{minted}{haskell}
interface Parser X Y =
    fail : Y
  | sat : {Char -> Bool} -> Char
  | choose : {[Parser X Y] Y} -> {[Parser X Y] Y} -> Y
  | many : {[Parser X Y] Y} -> List Y
\end{minted}
\end{block}
\end{frame}

\begin{frame}[fragile]{}
\begin{block}{Обработчик эффекта}
\begin{minted}{haskell}
runParser : (List Char) -> <Parser X Y>Y ->
            Pair (Maybe Y) (List Char)
runParser xs r = pair (just r) xs
runParser xs <fail -> _> = pair nothing xs
runParser nil <sat p -> k> = pair nothing nil
runParser (x::xs) <sat p -> k> =
  if (p x) {runParser xs (k x)} {pair nothing (x::xs)}
runParser xs <choose p1 p2 -> k> =
  on (runParser xs p1!)
    { (pair (just _) _) -> runParser xs p1!
    | (pair nothing _) -> on (runParser xs p2!)
      { (pair (just _) _) -> runParser xs p2!
      | (pair nothing _) -> pair nothing xs
      }
    }
\end{minted}
\end{block}
\end{frame}

\begin{frame}[fragile]{}
\begin{block}{Обработчик эффекта (продолжение)}
\begin{minted}{haskell}
runParser : (List Char) -> <Parser X Y>Y ->
            Pair (Maybe Y) (List Char)
runParser xs <many p -> k> =
  on (runParser xs p!)
      { (pair (just ok) rest) ->
          cons (pair (just ok) nil)
               (runParser rest (many p; k nil))
      | (pair nothing rest) -> runParser xs (k nil)
      }
\end{minted}
\end{block}
\end{frame}

\begin{frame}[fragile]{Проблема гомогенности}
\begin{block}{}
\begin{minted}{haskell}
runParser : (List Char) -> <Parser X Y>Y ->
            Pair (Maybe Y) (List Char)
\end{minted}
\end{block}
\pause
\begin{block}{}
% \begin{minted}{haskell}
$<Parser~Char~Char> \neq <Parser~Char~(List~Char)>$
% \end{minted}
\end{block}
\pause
\begin{block}{Гетерогенный список с помощью экзистенциального типа (\texttt{Haskell})}
\begin{minted}{haskell}
data Obj = forall a. (Show a) => Obj a

xs :: [Obj]
xs = [Obj 1, Obj "foo", Obj 'c']
\end{minted}
\end{block}
\end{frame}



\begin{frame}[fragile]{}
\begin{block}{Интерфейс эффекта}
\begin{minted}{haskell}
interface Parser =
    fail : forall Y . Y
  | sat : {Char -> Bool} -> Char
  | choose : forall Y . {[Parser] Y} -> {[Parser] Y} -> Y
  | many : forall Y . {[Parser] Y} -> List Y
\end{minted}
\end{block}
\end{frame}

% ∀

%%%%%%%%%%%%%%%%%%%%%%%%%%%%%%%%%%%%%%%%%%%%%%%%%%%%%%%%%%%%%%%%%%%%%%%%%%%%%%%%
%%%%%%%%%%%%%%%%%%%%%%%%%%%%  Conclusion  %%%%%%%%%%%%%%%%%%%%%%%%%%%%%%%%%%%%%%
%%%%%%%%%%%%%%%%%%%%%%%%%%%%%%%%%%%%%%%%%%%%%%%%%%%%%%%%%%%%%%%%%%%%%%%%%%%%%%%%

\section{Заключение}
\begin{frame}{Результаты}
  \begin{itemize}
    \item Построение парсеры на основе комбинации эффектов
    \item Исследование возможности описания парсера как монолитного эффекта
  \end{itemize}
\end{frame}

\begin{frame}[fragile]{Результаты}
\begin{itemize}
\item Всё
\item Хорошо
\end{itemize}
\end{frame}

%%%%%%%%%%%%%%%%%%%%%%%%%%%%%%%%%%%%%%%%%%%%%%%%%%%%%%%%%%%%%%%%%%%%%%%%%%%%%%%%
%%%%%%%%%%%%%%%%%%%%%%%%%%%%  References  %%%%%%%%%%%%%%%%%%%%%%%%%%%%%%%%%%%%%%
%%%%%%%%%%%%%%%%%%%%%%%%%%%%%%%%%%%%%%%%%%%%%%%%%%%%%%%%%%%%%%%%%%%%%%%%%%%%%%%%

\begin{frame}{References}

\begin{itemize}
  \item Do be do be do. Sam Lindley, Conor McBride, and Craig McLaughlin. POPL 2017.
  \item A. Bauer and M. Pretnar. Programming with algebraic effects and handlers. J. Log. Algebr. Meth. Program., 84(1):108–123, 2015. URL http://dx.doi.org/10.1016/j.jlamp.2014.02.001.
  \item P. B. Levy. Call-By-Push-Value: A Functional/Imperative Synthesis, volume 2 of Semantics Structures in Computation. Springer, 2004.
  \item Monadic Parser Combinators // \textit{Graham Hutton}, \textit{Erik Meijer} – Department of Computer Science, University of Nottingham, 1996
  \item Статья и эти слайды \url{https://github.com/geo2a/plc-mmcs-2017-frank-paper}
  \item Frankoparsec --- experimental parser combinators library implemented in Frank \url{https://github.com/geo2a/frankoparsec}
\end{itemize}

\end{frame}

\end{document}
