%% В этот файл не предполагается вносить изменения

% В этом файле следует указать информацию о себе
% и выполняемой работе.

\documentclass [fontsize=14pt, paper=a4, pagesize, DIV=calc]%
{scrreprt}
% ВНИМАНИЕ! Для использования глав поменять
% scrartcl на scrreprt

% Здесь ничего не менять
\usepackage [T2A] {fontenc}   % Кириллица в PDF файле
\usepackage [utf8] {inputenc} % Кодировка текста: utf-8
\usepackage [russian] {babel} % Переносы, лигатуры

%%%%%%%%%%%%%%%%%%%%%%%%%%%%%%%%%%%%%%%%%%%%%%%%%%%%%%%%%%%%%%%%%%%%%%%%
% Создание макроса управления элементами, специфичными
% для вида работы (курс., бак., маг.)
% Здесь ничего не менять:
\usepackage{ifthen}
\newcounter{worktype}
\newcommand{\typeOfWork}[1]
{
	\setcounter{worktype}{#1}
}

% ВНИМАНИЕ!
% Укажите тип работы: 0 - курсовая, 1 - бак., 2 - маг.,
% 3 - бакалаврская с главами.
\typeOfWork{2}
% Считается, что курсовая и бак. бьются на разделы (section) и
% подразделы (subsection), а маг. — на главы (chapter), разделы и
%  подразделы. Если хочется,
% чтобы бак. была с главами (например, если она большая),
% надо выбрать опцию 3.

% Если при выборе 2 или 3 вы забудете поменять класс
% документа на scrreprt (см. выше, в самом начале),
% то получите ошибку:
% ./aux/appearance.tex:52: Package scrbase Error: unknown option ` chapterprefix=

%%%%%%%%%%%%%%%%%%%%%%%%%%%%%%%%%%%%%%%%%%%%%%%%%%%%%%%%%%%%%%%%%%%%%%%%
% Информация об авторе и работе для титульной страницы

\usepackage {titling}

% Имя автора в именительном падеже (для маг.)
\newcommand {\me}{%
G.\,A.~Lukyanov%
}

% Имя автора в родительном падеже (для курсовой и бак.)
\newcommand {\byme}{%
И.\,И.~Иванова%
}

% Любимый научный руководитель
\newcommand{\supervisor}%
{учёная степень, учёное звание /  должность И. О. Фамилия}

% идентифицируем пол (только для курсовой и бак.)
\newcommand{\bystudent}{
Студента %Студентки % Для курсовой: с большой буквы
}

% Год публикации
\date{2017}

% Название работы
\title{Constructing effectful computations}

% Кафедра
%
\newboolean{needchair}
\setboolean{needchair}{false} % на ФИИТ не пишется (false), на ПМИ есть (true)

\newcommand {\thechair} {%
Кафедра компьютерного и аналогового моделирования светлого будущего%
}

\newcommand {\direction} {%
Направление подготовки\\
Фундаментальная информатика и информационные технологии%
}% Прикладная математика и информатика

%%%%%%%%%%%%%%%%%%%%%%%%%%%%%%%%%%%%%%%%%%%%%%%%%%%%%%%%%%%%%%%%%%%%%%%%
% Другие настраиваемые элементы текста

% Листинги с исходным кодом программ: укажите язык программирования
\usepackage{listings}
\lstset{
    language=Haskell,%  Язык указать здесь
    basicstyle=\small\ttfamily,
    breaklines=true,%
    showstringspaces=false%
    inputencoding=utf8x%
}
% полный список языков, поддерживаемых данным пакетом, есть,
% например, здесь (стр. 13):
% ftp://ftp.tex.ac.uk/tex-archive/macros/latex/contrib/listings/listings.pdf

% Нумерация списков: можно при необходимести
% изменять вид нумерации (например, добавлять правую скобку).
% По умолчанию буду списки вида:
% 1.
% 2.
% Изменять вид нумерации можно в начале нумерации:
% \begin{enumerate}[1)] (В квадратных скобках указан желаемый вид)
\usepackage[shortlabels]{enumitem}
                    \setlist[enumerate, 1]{1.}

% Гиперссылки: настройте внешний вид ссылок
\usepackage%
[pdftex,unicode,pdfborder={0 0 0},draft=false,%backref=page,
    hidelinks, % убрать, если хочется видеть ссылки: это
               % удобно в PDF файле, но не должно появиться на печати
    bookmarks=true,bookmarksnumbered=false,bookmarksopen=false]%
{hyperref}


\usepackage {amsmath}      % Больше математики
\usepackage {amssymb}
\usepackage {textcase}     % Преобразование к верхнему регистру
\usepackage {indentfirst}  % Красная строка первого абзаца в разделе
\usepackage [super]{nth}

\usepackage {fancyvrb}     % Листинги: определяем своё окружение Verb
\DefineVerbatimEnvironment% с уменьшенным шрифтом
	{Verb}{Verbatim}
	{fontsize=\small}

% Вставка рисунков
\usepackage {graphicx}

% Общее оформление
% ----------------------------------------------------------------
% Настройка внешнего вида

%%% Шрифты

% если закомментировать всё — консервативная гарнитура Computer Modern
\usepackage{paratype} % профессиональные свободные шрифты
%\usepackage {droid}  % неплохие свободные шрифты от Google
%\usepackage{mathptmx}
%\usepackage {mmasym}
%\usepackage {psfonts}
%\usepackage{lmodern}
%var1: lh additions for bold concrete fonts
%\usepackage{lh-t2axccr}
%var2: the package below could be covered with fd-files
%\usepackage{lh-t2accr}
%\usepackage {pscyr}

% Геометрия текста

\usepackage{setspace}       % Межстрочный интервал
\onehalfspacing

\newlength\MyIndent
\setlength\MyIndent{1.25cm}
\setlength{\parindent}{\MyIndent} % Абзацный отступ
\frenchspacing            % Отключение лишних отступов после точек
\KOMAoptions{%
    DIV=calc,         % Пересчёт геометрии
    numbers=endperiod % точки после номеров разделов
}

                            % Консервативный вариант:
%\usepackage                % ручное задание геометрии
%[%                         % (не рекомендуется в проф. типографии)
%  margin = 2.5cm,
  %includefoot,
  %footskip = 1cm
%] %
%  {geometry}

%%% Заголовки

\ifthenelse{\equal{\theworktype}{2}}{%
\KOMAoptions{%
    numbers=endperiod,% точки после номеров разделов
    headings=normal,   % размеры заголовков поменьше стандартных
    chapterprefix=true,% Печатать слово Глава в магистерской
    appendixprefix=true% Печатать слово Приложение
}
}

% шрифт для оформления глав и названия содержания
\newcommand{\SuperFont}{\Large\sffamily\bfseries}

% Заголовок главы
\ifthenelse{\value{worktype} > 1}{%
\renewcommand{\SuperFont}{\Large\normalfont\sffamily}
\newcommand{\CentSuperFont}{\centering\SuperFont}
\usepackage{fncychap}
\ChNameVar{\SuperFont}
\ChNumVar{\CentSuperFont}
\ChTitleVar{\CentSuperFont}
\ChNameUpperCase
\ChTitleUpperCase
}

% Заголовок (под)раздела с абзацного отступа
\addtokomafont{sectioning}{\hspace{\MyIndent}}

\renewcommand*{\captionformat}{~---~}
\renewcommand*{\figureformat}{Listing~\thefigure}

% Плавающие листинги
\usepackage{float}
\floatstyle{ruled}
\floatname{ListingEnv}{Листинг}
\newfloat{ListingEnv}{htbp}{lol}[section]

% точка после номера листинга
\makeatletter
\renewcommand\floatc@ruled[2]{{\@fs@cfont #1.} #2\par}
\makeatother


%%% Оглавление
\usepackage{tocloft}

% шрифт и положение заголовка
\ifthenelse{\value{worktype} > 1}{%
\renewcommand{\cfttoctitlefont}{\hfil\SuperFont\MakeUppercase}
}{
\renewcommand{\cfttoctitlefont}{\hfil\SuperFont}
}

% слово Глава
\usepackage{calc}
\ifthenelse{\value{worktype} > 1}{%
\renewcommand{\cftchappresnum}{Chapter }
\addtolength{\cftchapnumwidth}{\widthof{Chapter }}
}

\newcommand{\setupname}[1]{%
  \addtocontents{toc}{%
    \unexpanded{\unexpanded{%
      \renewcommand{\cftchappresnum}{#1 }%
      \setlength\cftchapnumwidth{\widthof{\bfseries #1 }}%
      \addtolength\cftchapnumwidth{\fixedchapnumwidth}%
    }}%
  }%
}
\AtBeginDocument{\edef\fixedchapnumwidth{\the\cftchapnumwidth}}

% Очищаем оформление названий старших элементов в оглавлении
\ifthenelse{\value{worktype} > 1}{%
\renewcommand{\cftchapfont}{}
\renewcommand{\cftchappagefont}{}
}{
\renewcommand{\cftsecfont}{}
\renewcommand{\cftsecpagefont}{}
}

% Точки после верхних элементов оглавления
\renewcommand{\cftsecdotsep}{\cftdotsep}
%\newcommand{\cftchapdotsep}{\cftdotsep}

\ifthenelse{\value{worktype} > 1}{%
    \renewcommand{\cftchapaftersnum}{.}
}{}
\renewcommand{\cftsecaftersnum}{.}
\renewcommand{\cftsubsecaftersnum}{.}
\renewcommand{\cftsubsubsecaftersnum}{.}

%%% Списки (enumitem)

\usepackage {enumitem}      % Списки с настройкой отступов
\setlist %
{ %
  leftmargin = \parindent, itemsep=.5ex, topsep=.4ex
} %

% По ГОСТу нумерация должны быть буквами: а, б...
%\makeatletter
%    \AddEnumerateCounter{\asbuk}{\@asbuk}{м)}
%\makeatother
%\renewcommand{\labelenumi}{\asbuk{enumi})}
%\renewcommand{\labelenumii}{\arabic{enumii})}

%%% Таблицы: выбрать более подходящие

\usepackage{booktabs} % считаются наиболее профессионально выполненными
%\usepackage{ltablex}
%\newcolumntype {L} {>{---}l}

%%% Библиография

\usepackage{csquotes}        % Оформление списка литературы
\usepackage[
  backend=biber,
  hyperref=auto,
  sorting=none, % сортировка в порядке встречаемости ссылок
  language=auto,
  citestyle=gost-numeric,
  bibstyle=gost-numeric
]{biblatex}
\addbibresource{biblio.bib} % Файл с лит.источниками

% Настройка величины отступа в списке
\ifthenelse{\value{worktype} < 2}{%
\defbibenvironment{bibliography}
  {\list
     {\printtext[labelnumberwidth]{%
    \printfield{prefixnumber}%
    \printfield{labelnumber}}}
     {\setlength{\labelwidth}{\labelnumberwidth}%
      \setlength{\leftmargin}{\labelwidth}%
      \setlength{\labelsep}{\dimexpr\MyIndent-\labelwidth\relax}% <----- default is \biblabelsep
      \addtolength{\leftmargin}{\labelsep}%
      \setlength{\itemsep}{\bibitemsep}%
      \setlength{\parsep}{\bibparsep}}%
      \renewcommand*{\makelabel}[1]{\hss##1}}
  {\endlist}
  {\item}
}{}

% ----------------------------------------------------------------
% Настройка переносов и разрывов страниц

\binoppenalty = 10000      % Запрет переносов строк в формулах
\relpenalty = 10000        %

\sloppy                    % Не выходить за границы бокса
%\tolerance = 400          % или более точно
\clubpenalty = 10000       % Запрет разрывов страниц после первой
\widowpenalty = 10000      % и перед предпоследней строкой абзаца

% ----------------------------


% Стили для окружений типа Определение, Теорема...
% Оформление теорем (ntheorem)

\usepackage [thmmarks, amsmath] {ntheorem}
\theorempreskipamount 0.6cm

\theoremstyle {plain} %
\theoremheaderfont {\normalfont \bfseries} %
\theorembodyfont {\slshape} %
\theoremsymbol {\ensuremath {_\Box}} %
\theoremseparator {:} %
\newtheorem {mystatement} {Утверждение} [section] %
\newtheorem {mylemma} {Лемма} [section] %
\newtheorem {mycorollary} {Следствие} [section] %

\theoremstyle {nonumberplain} %
\theoremseparator {.} %
\theoremsymbol {\ensuremath {_\diamondsuit}} %
\newtheorem {mydefinition} {Определение} %

\theoremstyle {plain} %
\theoremheaderfont {\normalfont \bfseries} 
\theorembodyfont {\normalfont} 
%\theoremsymbol {\ensuremath {_\Box}} %
\theoremseparator {.} %
\newtheorem {mytask} {Задача} [section]%
\renewcommand{\themytask}{\arabic{mytask}}

\theoremheaderfont {\scshape} %
\theorembodyfont {\upshape} %
\theoremstyle {nonumberplain} %
\theoremseparator {} %
\theoremsymbol {\rule {1ex} {1ex}} %
\newtheorem {myproof} {Доказательство} %

\theorembodyfont {\upshape} %
%\theoremindent 0.5cm
\theoremstyle {nonumberbreak} \theoremseparator {\\} %
\theoremsymbol {\ensuremath {\ast}} %
\newtheorem {myexample} {Пример} %
\newtheorem {myexamples} {Примеры} %

\theoremheaderfont {\itshape} %
\theorembodyfont {\upshape} %
\theoremstyle {nonumberplain} %
\theoremseparator {:} %
\theoremsymbol {\ensuremath {_\triangle}} %
\newtheorem {myremark} {Замечание} %
\theoremstyle {nonumberbreak} %
\newtheorem {myremarks} {Замечания} %


% Титульный лист
% Макросы настройки титульной страницы
% В этот файл не предполагается вносить изменения

%\usepackage {showframe}

% Вертикальные отступы на титульной странице
\newcommand{\vgap}{\vspace{16pt}}

% Помещение города и даты в нижний колонтитул
\usepackage{scrlayer}
\DeclareNewLayer[
  foot,
  foreground,
  contents={%
    \raisebox{\dp\strutbox}[\layerheight][0pt]{%
      \parbox[b]{\layerwidth}{\centering Ростов-на-Дону\\ \thedate%
       \\\mbox{}
       }}%
  }
]{titlepage.foot.fg}
\DeclareNewPageStyleByLayers{titlepage}{titlepage.foot.fg}


\AtBeginDocument %
{ %
  %
  \begin{titlepage}
  %
    \thispagestyle{titlepage}

    {\centering
    %
    \MakeTextUppercase {МИНИСТЕРСТВО ОБРАЗОВАНИЯ И НАУКИ РФ}

    \vgap

    Федеральное государственное автономное образовательное\\
    учреждение высшего образования\\
    \MakeTextUppercase {Южный федеральный университет}

    \vgap

	Институт математики, механики и компьютерных наук
    имени~И.\,И.\,Воровича

    \vgap

    \direction

    \ifthenelse{\boolean{needchair}}{
    \vgap

    \thechair}{}

    \vspace* {\fill}

    \ifthenelse{\value{worktype} = 2}{%
    \me

    \vgap}{}

    {\usefont{T2A}{PTSansCaption-TLF}{m}{n}
    \MakeTextUppercase{\thetitle}}

    \ifthenelse{\value{worktype} = 2}{%
     \vgap

    Магистерская диссертация}{}
    \ifthenelse{\value{worktype} = 0}{
     \vgap

    Курсовая работа
    }{}%
    \ifthenelse{\value{worktype} = 1 \OR \value{worktype} = 3}{
     \vgap

    Выпускная квалификационная работа\\
    на степень бакалавра
    }{}%

    \vspace {\fill}

    \begin{flushright}
    \ifthenelse{\value{worktype} = 0 \OR
                \value{worktype} = 1 \OR
                \value{worktype} = 3}{
      \bystudent \ifthenelse{\value{worktype} = 0}{3}{4}\ курса\\
      \byme
    }{}

    \vgap

    Научный руководитель:\\
    к.т.н., с.н.с., доц. каф. ИВЭ
    А.~Н.~Литвиненко\\
    \ifthenelse{\value{worktype} = 2}{%
    Рецензент:\\
    к.ф.-м.н., доц., доц. каф. АДМ С.~С.~Михалкович}{}
	\end{flushright}
    \ifthenelse{\value{worktype} = 0}{
    \vspace{\fill}
            \begin{flushleft}
              \begin{tabular}{cc}
                \underline{\hspace{4cm}}&\underline{\hspace{5cm}}\\
                {\small оценка (рейтинг)} & {\small  подпись руководителя}\\
              \end{tabular}
            \end{flushleft}
    }{}
  	\vspace {\fill}
  %Ростов-на-Дону

    %\thedate

  }\end{titlepage}
  %
  %
  \tableofcontents
  %
  \clearpage
} %



% Команды для использования в тексте работы


% макросы для начала введения и заключения
\newcommand{\Ackns}{\addchap{Acknowledgements}}

\newcommand{\Intro}{\addchap{Introduction}}

\newcommand{\Goal}{\addchap{Goal statement}}

\newcommand{\Conc}{\addchap{Conclusion}}

% Правильные значки для нестрогих неравенств и пустого множества
\renewcommand {\le} {\leqslant}
\renewcommand {\ge} {\geqslant}
\renewcommand {\emptyset} {\varnothing}

% N ажурное: натуральные числа
\newcommand {\N} {\ensuremath{\mathbb N}}

% значок С++ — используйте команду \cpp
\newcommand{\cpp}{%
C\nolinebreak\hspace{-.05em}%
\raisebox{.2ex}{+}\nolinebreak\hspace{-.10em}%
\raisebox{.2ex}{+}%
}

% значок С# — используйте команду \cs
\newcommand{\cs}{%
C\nolinebreak\hspace{-.05em}%
\raisebox{.2ex}{\#}%
}

% Неразрывный дефис, который допускает перенос внутри слов,
% типа жёлто-синий: нужно писать жёлто"/синий.
\makeatletter
  \defineshorthand[english]{"/}{\babelhyphen{nobreak}}
  \addto\extrasenglish{
    \languageshorthands{english}
    \useshorthands{"}
  }
\makeatother



\endinput

% Конец файла


\usepackage[cache=false]{minted}

\title[]{Структурирование вычислений с эффектами}

\subtitle{}

% \author[Георгий Лукьянов]{%
% Георгий Лукьянов\texorpdfstring{\\}{ }
% georgiy.lukjanov@gmail.com}

% \author[]{%
% Георгий Лукьянов\texorpdfstring{\\}{ }
% \textit{georgiy.lukjanov@gmail.com}\texorpdfstring{\\}{ }
% Артём Пеленицын \texorpdfstring{\\}{ }
% \textit{apel@sfedu.ru}
% }


\author[Г.\,А.\,Лукьянов \& А.\,М.\,Пеленицын]
{%
  % \texorpdfstring{
  %   \begin{columns}
  %     % \column{.45\linewidth}
  %     % \centering
  %     % John Doe\          \href{mailto:john@example.com}{john@example.com}
  %     % \column{.45\linewidth}
  %     % \centering
  %     % Jane Doe\          \href{mailto:jane.doe@example.com}{jane.doe@example.com}
  %     \column{.45\linewidth}
  %     \centering
  %     Георгий Лукьянов\\ \scriptsize{\textit{georgiy.lukjanov@gmail.com}}
  %     \column{.45\linewidth}
  %     \centering
  %     Артём Пеленицын\\ \scriptsize{\textit{apel@sfedu.ru}}
  %   \end{columns}
  % }
  {Г.\,А.\,Лукьянов \& А.\,М.\,Пеленицын}
}

\date{21 Июня 2017}%

\institute[]{%
Южный Федеральный Университет \texorpdfstring{\\}{ }
Институт Математики, Механики и Компьютерных Наук имени~И.\,И.\,Воровича\texorpdfstring{\\}{ }
}

\begin{document}

\begin{frame}
\titlepage
\end{frame}

%%%%%%%%%%%%%%%%%%%%%%%%%%%%%%%%%%%%%%%%%%%%%%%%%%%%%%%%%%%%%%%%%%%%%%%%%%%%%%%%
%%%%%%%%%%%%%%%%%  Intro  %%%%%%%%%%%%%%%%%%%%%%%%%%%%%%%%%%%%%%%%%%%%%%%%%%%%%%
%%%%%%%%%%%%%%%%%%%%%%%%%%%%%%%%%%%%%%%%%%%%%%%%%%%%%%%%%%%%%%%%%%%%%%%%%%%%%%%%

\begin{frame}[fragile]{Постановка задачи}
\begin{itemize}
\item Разработать библиотеку монадических комбинаторов парсеров для
иллюстрации использования преобразователей монад в Haskell.
\item Разработать библиотеку комбинаторов парсеров на основе расширяемых эффектов~---
реализации концепции расширяемых эффектов в Haskell.
\item Выделить отличия преобразователей монад и расширяемых эффектов.
\item Проанализировать возможности экспериментального языка программирования Frank,
поддерживающего концепцию алгебраических эффектов на уровне ядра, для реализации
библиотек комбинаторов парсеров.
\item Произвести анализ производительности разработанных библиотек и сравнить с
популярными аналогами.
\item Продемонстрировать преимущества типизированного функционального языка программирования
с системой эффектов при реализации крупных систем: разработать полномасштабное
веб-приложение на Haskell.
\end{itemize}
\end{frame}

\section{Управление побочными эффектами}

% \begin{frame}{Подходы к управлению побочными эффектами (не является классификацией)}
% \begin{itemize}
%   \item Функторы
%   \item Аппликативные функторы
%   \item Монады (И трансформеры монад)
%   \item Стрелки
%   \item Алгебраические эффекты и обработчики эффектов
% \end{itemize}
% \end{frame}

\begin{frame}[fragile]{Вычислительные эффекты}
\begin{columns}
\begin{column}{0.55\textwidth}
  \begin{block}{}
  \begin{minted}{haskell}
  add : Int -> Int -> Int
  \end{minted}
  \end{block}
  \begin{block}{}
  \begin{minted}{haskell}
  addSt : State (Int, Int) Int
  \end{minted}
  \end{block}
  \begin{block}{}
  \begin{minted}{haskell}
  addIO : IO Int
  \end{minted}
  \end{block}
\end{column}
\begin{column}{0.4\textwidth}
  \begin{itemize}
    \item Контролируемые
    \begin{itemize}
      \item \texttt{State}
      \item \texttt{Reader}
      \item \texttt{Exception}
      \item \texttt{...}
    \end{itemize}
    \item Неконтролируемые
    \begin{itemize}
      \item \texttt{IO}
    \end{itemize}
  \end{itemize}
\end{column}
\end{columns}
\end{frame}

%%%%%%%%%%%%%%%%%%%%%%%%%%%%%%%%%%%%%%%%%%%%%%%%%%%%%%%%%%%%%%%%%%%%%%%%%%%%%%%%
%%%%%%%%%%%%%%%%%  Frank  %%%%%%%%%%%%%%%%%%%%%%%%%%%%%%%%%%%%%%%%%%%%%%%%%%%%%%
%%%%%%%%%%%%%%%%%%%%%%%%%%%%%%%%%%%%%%%%%%%%%%%%%%%%%%%%%%%%%%%%%%%%%%%%%%%%%%%%

\section{Язык программирования Frank}

\begin{frame}{Язык программирования Frank}
  Последняя версия представлена на POPL'17 в статье Do be do be do от Sam Lindley, Conor McBride, Craig McLaughlin
  \begin{block}{Особенности}
    \begin{itemize}
      \item \textbf{Строгая} стратегия вычислений
      \item Алгебраические типы данных
      \item Разделение \textbf{типов-значений} и \textbf{типов-вычислений}
      \item \textbf{Интерфейсы} --- сигнатуры эффектов
      \item \textbf{Операции} --- обработчики эффектов
      \item Обычные функции являются тривиальными операторами, обрабатывающими пустой множество эффектов
      \item Полиморфизм эффектов основанный на понятии \textbf{охватывающего поля эффектов} (англ. ambient ability)
    \end{itemize}
  \end{block}
\end{frame}

\subsection{Базовые конструкции}

\begin{frame}[fragile]{}
\begin{block}{Натуральные числа}
\begin{minted}{haskell}
data Nat = zero | suc Nat
\end{minted}
\end{block}
\begin{block}{Рекурсивные функции}
\begin{minted}{haskell}
eqNat : Nat -> Nat -> Bool
eqNat zero zero  = true
eqNat (suc x) (suc y) = eqNat x y
eqNat zero _     = false
eqNat _  zero    = false
\end{minted}
\end{block}
\end{frame}

\subsection{Типы-значения и типы-вычисления}

\begin{frame}[fragile]{}
\begin{block}{Вычисляющая условная операция}
\begin{minted}{haskell}
iffi : Bool -> X -> X -> X
iffi true  t f = t
iffi false t f = f
\end{minted}
\end{block}
\begin{block}{Стандартная условная операция}
\begin{minted}{haskell}
if : Bool -> {X} -> {X} -> X
if true  t f = t!
if false t f = f!
\end{minted}
\end{block}
\end{frame}

% \begin{frame}[fragile]{}
% \begin{block}{Функция \texttt{map} и отложенные вычисления}
% \begin{minted}{haskell}
% map : {X -> Y} -> List X -> List Y
% map f nil         = nil
% map f (cons x xs) = cons (f x) (map f xs)
% \end{minted}
% \end{block}
% \begin{block}{Применение \texttt{map} в отсутствие эффектов}
% \begin{minted}{haskell}
% pureMap : {List Int}
% pureMap = map {x -> x + 1} [1,2,3]
% \end{minted}
% \end{block}
% \end{frame}

\subsection{Контроль побочных эффектов}

\begin{frame}[fragile]{Интерфейсы эффектов}
\begin{block}{Эффект изменяемого состояния}
\begin{minted}{haskell}
interface State S = get : S
                  | put : S -> Unit
\end{minted}
\end{block}
\begin{block}{Эффект потенциально ошибочных вычислений}
\begin{minted}{haskell}
interface Error = fail : Zero
\end{minted}
\end{block}
\end{frame}

\begin{frame}[fragile]{Обработчики эффектов}
\begin{block}{Обработчик эффекта \texttt{State}}
\begin{minted}{haskell}
state : S -> <State S>X -> X
state _ x             = x
state s <get -> k>    = state s (k s)
state _ <put s -> k>  = state s (k unit)
\end{minted}
\end{block}
\begin{block}{Обработчик эффекта \texttt{Error}}
\begin{minted}{haskell}
catch : <Error>X -> Maybe X
catch x = just x
catch <fail -> _> = nothing
\end{minted}
\end{block}
\end{frame}



% \begin{frame}[fragile]{Управление потоком выполнения}
% \begin{block}{Энергичный условный оператор}
% \begin{minted}{haskell}
% iffy : Bool -> X -> X -> X
% iffy tt t f = t
% iffy ff t f = f
% \end{minted}
% \end{block}
% \begin{block}{``Традиционный'' условный оператор}
% \begin{minted}{haskell}
% if : Bool -> {X} -> {X} -> X
% if tt t f = t!
% if ff t f = f!
% \end{minted}
% \end{block}
% \end{frame}

% \begin{frame}[fragile]{Эффект изменяемого состояния}
% \begin{block}{Интерфейс эффекта \texttt{State}}
% \begin{minted}{haskell}
% interface State S = get : S
%                   | put : S -> Unit
% \end{minted}
% \end{block}
% \begin{block}{Обработчик эффекта \texttt{State}}
% \begin{minted}{haskell}
% state : S -> <State S>X -> X
% state _ x          = x
% state s <get -> k>  = state s (k s)
% state _ <put s -> k>  = state s (k unit)
% \end{minted}
% \end{block}
% \end{frame}

% \begin{frame}[fragile]{Исключения}
% \begin{block}{Интерфейс эффекта \texttt{Exception}}
% \begin{minted}{haskell}
% interface Exception E
%   = exception : E -> Zero
% \end{minted}
% \end{block}
% \begin{block}{Возбуждение исключения}
% \begin{minted}{haskell}
% throw : E -> [Exception E]X
% throw e = on (exception e) {}
% \end{minted}
% \end{block}
% \begin{block}{Обработка исключения}
% \begin{minted}{haskell}
% catch : <Exception E>X -> Either E X
% catch x = right x
% catch <exception e -> _> = left e
% \end{minted}
% \end{block}

% \end{frame}





%%%%%%%%%%%%%%%%%%%%%%%%%%%%%%%%%%%%%%%%%%%%%%%%%%%%%%%%%%%%%%%%%%%%%%%%%%%%%%%%
%%%%%%%%%%%%%  Building Parsers Combinators with Frank  %%%%%%%%%%%%%%%%%%%%%%%%
%%%%%%%%%%%%%%%%%%%%%%%%%%%%%%%%%%%%%%%%%%%%%%%%%%%%%%%%%%%%%%%%%%%%%%%%%%%%%%%%

\section{Построение парсеров на Frank}

\subsection{Парсер как комбинация эффектов}

\begin{frame}[fragile]{Парсер как комбинация эффектов}

\begin{block}{Обработка комбинации эффектов}
\begin{minted}{haskell}
parse : {[Error, State (List Char)] X} ->
        (List Char) -> Maybe X
parse p str = catch (state str p!)
\end{minted}
\end{block}

\pause

\begin{block}{Базовые парсеры}
\begin{minted}{haskell}
item : [Error, State (List Char)] Char
item! = on get! { nil       -> fail
                | (x :: xs) -> put xs; x}

sat : {Char -> [Error, State (List Char)] Bool} ->
      [Error, State (List Char)] Char
sat p = on item! {c -> if (p c) {c} {fail}}
\end{minted}
\end{block}
\end{frame}

\begin{frame}[fragile]{Парсер как комбинация эффектов}
\begin{block}{Парсер для заданного символа}
\begin{minted}{haskell}
char : Char -> [Error, State (List Char)] Char
char c = sat {x -> eqChar x c}
\end{minted}
\end{block}

\begin{block}{Парсер для заданной строки}
\begin{minted}{haskell}
string : (List Char) ->
         [Error, State (List Char)] (List Char)
string str = map char str
\end{minted}
\end{block}
\end{frame}

\begin{frame}[fragile]{Парсер как комбинация эффектов}
\begin{block}{Комбинатор детерминированного выбора}
\begin{minted}{haskell}
choose : {[Error, State (List Char)] X} ->
         {[Error, State (List Char)] X} ->
          [Error, State (List Char)] X
choose p1 p2 =
  on (parse p1 get!) { (right _)  -> p1!
                     | (left  _)  ->  on (parse p2 get!)
                       { (right _)   -> p2!
                       | (left err)  -> fail
                       }
                     }
\end{minted}
\end{block}
\end{frame}

\begin{frame}[fragile]{Парсер как комбинация эффектов}
\begin{block}{Комбинаторы последовательности}
\begin{minted}{haskell}
many : {[Error, State (List Char)]X} ->
        [Error, State (List Char)](List X)
many p = choose {some p} {nil}

some : {[Error, State (List Char)] X} ->
        [Error, State (List Char)](List X)
some p = p! :: many p
\end{minted}
\end{block}
\end{frame}





\subsection{Парсер как монолитный эффект}

\begin{frame}[fragile]{}
\begin{block}{Интерфейс эффекта}
\begin{minted}{haskell}
interface Parser X Y =
    fail : Y
  | sat : {Char -> Bool} -> Char
  | choose : {[Parser X Y] Y} -> {[Parser X Y] Y} -> Y
  | many : {[Parser X Y] Y} -> List Y
\end{minted}
\end{block}
\end{frame}

\begin{frame}[fragile]{}
\begin{block}{Обработчик эффекта}
\begin{minted}{haskell}
runParser : (List Char) -> <Parser X Y>Y ->
            Pair (Maybe Y) (List Char)
runParser xs r = pair (just r) xs
runParser xs <fail -> _> = pair nothing xs
runParser nil <sat p -> k> = pair nothing nil
runParser (x::xs) <sat p -> k> =
  if (p x) {runParser xs (k x)} {pair nothing (x::xs)}
runParser xs <choose p1 p2 -> k> =
  on (runParser xs p1!)
    { (pair (just _) _) -> runParser xs p1!
    | (pair nothing _) -> on (runParser xs p2!)
      { (pair (just _) _) -> runParser xs p2!
      | (pair nothing _) -> pair nothing xs
      }
    }
\end{minted}
\end{block}
\end{frame}

\begin{frame}[fragile]{}
\begin{block}{Обработчик эффекта (продолжение)}
\begin{minted}{haskell}
runParser : (List Char) -> <Parser X Y>Y ->
            Pair (Maybe Y) (List Char)
runParser xs <many p -> k> =
  on (runParser xs p!)
      { (pair (just ok) rest) ->
          cons (pair (just ok) nil)
               (runParser rest (many p; k nil))
      | (pair nothing rest) -> runParser xs (k nil)
      }
\end{minted}
\end{block}
\end{frame}

\begin{frame}[fragile]{Проблема гомогенности}
\begin{block}{}
\begin{minted}{haskell}
runParser : (List Char) -> <Parser X Y>Y ->
            Pair (Maybe Y) (List Char)
\end{minted}
\end{block}
\pause
\begin{block}{}
% \begin{minted}{haskell}
$<Parser~Char~Char> \neq <Parser~Char~(List~Char)>$
% \end{minted}
\end{block}
\pause
\begin{block}{Гетерогенный список с помощью экзистенциального типа (\texttt{Haskell})}
\begin{minted}{haskell}
data Obj = forall a. (Show a) => Obj a

xs :: [Obj]
xs = [Obj 1, Obj "foo", Obj 'c']
\end{minted}
\end{block}
\end{frame}



\begin{frame}[fragile]{}
\begin{block}{Интерфейс эффекта}
\begin{minted}{haskell}
interface Parser =
    fail : forall Y . Y
  | sat : {Char -> Bool} -> Char
  | choose : forall Y . {[Parser] Y} -> {[Parser] Y} -> Y
  | many : forall Y . {[Parser] Y} -> List Y
\end{minted}
\end{block}
\end{frame}

% ∀

%%%%%%%%%%%%%%%%%%%%%%%%%%%%%%%%%%%%%%%%%%%%%%%%%%%%%%%%%%%%%%%%%%%%%%%%%%%%%%%%
%%%%%%%%%%%%%%%%%%%%%%%%%%%%  Conclusion  %%%%%%%%%%%%%%%%%%%%%%%%%%%%%%%%%%%%%%
%%%%%%%%%%%%%%%%%%%%%%%%%%%%%%%%%%%%%%%%%%%%%%%%%%%%%%%%%%%%%%%%%%%%%%%%%%%%%%%%

\section{Заключение}
\begin{frame}{Результаты}
  \begin{itemize}
    \item Построение парсеры на основе комбинации эффектов
    \item Исследование возможности описания парсера как монолитного эффекта
  \end{itemize}
\end{frame}

\begin{frame}[fragile]{Результаты}
\begin{itemize}
\item Всё
\item Хорошо
\end{itemize}
\end{frame}

%%%%%%%%%%%%%%%%%%%%%%%%%%%%%%%%%%%%%%%%%%%%%%%%%%%%%%%%%%%%%%%%%%%%%%%%%%%%%%%%
%%%%%%%%%%%%%%%%%%%%%%%%%%%%  References  %%%%%%%%%%%%%%%%%%%%%%%%%%%%%%%%%%%%%%
%%%%%%%%%%%%%%%%%%%%%%%%%%%%%%%%%%%%%%%%%%%%%%%%%%%%%%%%%%%%%%%%%%%%%%%%%%%%%%%%

\begin{frame}{References}

\begin{itemize}
  \item Do be do be do. Sam Lindley, Conor McBride, and Craig McLaughlin. POPL 2017.
  \item A. Bauer and M. Pretnar. Programming with algebraic effects and handlers. J. Log. Algebr. Meth. Program., 84(1):108–123, 2015. URL http://dx.doi.org/10.1016/j.jlamp.2014.02.001.
  \item P. B. Levy. Call-By-Push-Value: A Functional/Imperative Synthesis, volume 2 of Semantics Structures in Computation. Springer, 2004.
  \item Monadic Parser Combinators // \textit{Graham Hutton}, \textit{Erik Meijer} – Department of Computer Science, University of Nottingham, 1996
  \item Статья и эти слайды \url{https://github.com/geo2a/plc-mmcs-2017-frank-paper}
  \item Frankoparsec --- experimental parser combinators library implemented in Frank \url{https://github.com/geo2a/frankoparsec}
\end{itemize}

\end{frame}

\end{document}
