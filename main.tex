% В этом файле следует писать текст работы, разбивая его на
% разделы (section), подразделы (subsection) и, если нужно,
% главы (chapter).

% Предварительно следует указать необходимую информацию
% в файле SETUP.tex

%% В этот файл не предполагается вносить изменения

% В этом файле следует указать информацию о себе
% и выполняемой работе.

\documentclass [fontsize=14pt, paper=a4, pagesize, DIV=calc]%
{scrreprt}
% ВНИМАНИЕ! Для использования глав поменять
% scrartcl на scrreprt

% Здесь ничего не менять
\usepackage [T2A] {fontenc}   % Кириллица в PDF файле
\usepackage [utf8] {inputenc} % Кодировка текста: utf-8
\usepackage [russian] {babel} % Переносы, лигатуры

%%%%%%%%%%%%%%%%%%%%%%%%%%%%%%%%%%%%%%%%%%%%%%%%%%%%%%%%%%%%%%%%%%%%%%%%
% Создание макроса управления элементами, специфичными
% для вида работы (курс., бак., маг.)
% Здесь ничего не менять:
\usepackage{ifthen}
\newcounter{worktype}
\newcommand{\typeOfWork}[1]
{
	\setcounter{worktype}{#1}
}

% ВНИМАНИЕ!
% Укажите тип работы: 0 - курсовая, 1 - бак., 2 - маг.,
% 3 - бакалаврская с главами.
\typeOfWork{2}
% Считается, что курсовая и бак. бьются на разделы (section) и
% подразделы (subsection), а маг. — на главы (chapter), разделы и
%  подразделы. Если хочется,
% чтобы бак. была с главами (например, если она большая),
% надо выбрать опцию 3.

% Если при выборе 2 или 3 вы забудете поменять класс
% документа на scrreprt (см. выше, в самом начале),
% то получите ошибку:
% ./aux/appearance.tex:52: Package scrbase Error: unknown option ` chapterprefix=

%%%%%%%%%%%%%%%%%%%%%%%%%%%%%%%%%%%%%%%%%%%%%%%%%%%%%%%%%%%%%%%%%%%%%%%%
% Информация об авторе и работе для титульной страницы

\usepackage {titling}

% Имя автора в именительном падеже (для маг.)
\newcommand {\me}{%
G.\,A.~Lukyanov%
}

% Имя автора в родительном падеже (для курсовой и бак.)
\newcommand {\byme}{%
И.\,И.~Иванова%
}

% Любимый научный руководитель
\newcommand{\supervisor}%
{учёная степень, учёное звание /  должность И. О. Фамилия}

% идентифицируем пол (только для курсовой и бак.)
\newcommand{\bystudent}{
Студента %Студентки % Для курсовой: с большой буквы
}

% Год публикации
\date{2017}

% Название работы
\title{Constructing effectful computations}

% Кафедра
%
\newboolean{needchair}
\setboolean{needchair}{false} % на ФИИТ не пишется (false), на ПМИ есть (true)

\newcommand {\thechair} {%
Кафедра компьютерного и аналогового моделирования светлого будущего%
}

\newcommand {\direction} {%
Направление подготовки\\
Фундаментальная информатика и информационные технологии%
}% Прикладная математика и информатика

%%%%%%%%%%%%%%%%%%%%%%%%%%%%%%%%%%%%%%%%%%%%%%%%%%%%%%%%%%%%%%%%%%%%%%%%
% Другие настраиваемые элементы текста

% Листинги с исходным кодом программ: укажите язык программирования
\usepackage{listings}
\lstset{
    language=Haskell,%  Язык указать здесь
    basicstyle=\small\ttfamily,
    breaklines=true,%
    showstringspaces=false%
    inputencoding=utf8x%
}
% полный список языков, поддерживаемых данным пакетом, есть,
% например, здесь (стр. 13):
% ftp://ftp.tex.ac.uk/tex-archive/macros/latex/contrib/listings/listings.pdf

% Нумерация списков: можно при необходимести
% изменять вид нумерации (например, добавлять правую скобку).
% По умолчанию буду списки вида:
% 1.
% 2.
% Изменять вид нумерации можно в начале нумерации:
% \begin{enumerate}[1)] (В квадратных скобках указан желаемый вид)
\usepackage[shortlabels]{enumitem}
                    \setlist[enumerate, 1]{1.}

% Гиперссылки: настройте внешний вид ссылок
\usepackage%
[pdftex,unicode,pdfborder={0 0 0},draft=false,%backref=page,
    hidelinks, % убрать, если хочется видеть ссылки: это
               % удобно в PDF файле, но не должно появиться на печати
    bookmarks=true,bookmarksnumbered=false,bookmarksopen=false]%
{hyperref}


\usepackage {amsmath}      % Больше математики
\usepackage {amssymb}
\usepackage {textcase}     % Преобразование к верхнему регистру
\usepackage {indentfirst}  % Красная строка первого абзаца в разделе
\usepackage [super]{nth}

\usepackage {fancyvrb}     % Листинги: определяем своё окружение Verb
\DefineVerbatimEnvironment% с уменьшенным шрифтом
	{Verb}{Verbatim}
	{fontsize=\small}

% Вставка рисунков
\usepackage {graphicx}

% Общее оформление
% ----------------------------------------------------------------
% Настройка внешнего вида

%%% Шрифты

% если закомментировать всё — консервативная гарнитура Computer Modern
\usepackage{paratype} % профессиональные свободные шрифты
%\usepackage {droid}  % неплохие свободные шрифты от Google
%\usepackage{mathptmx}
%\usepackage {mmasym}
%\usepackage {psfonts}
%\usepackage{lmodern}
%var1: lh additions for bold concrete fonts
%\usepackage{lh-t2axccr}
%var2: the package below could be covered with fd-files
%\usepackage{lh-t2accr}
%\usepackage {pscyr}

% Геометрия текста

\usepackage{setspace}       % Межстрочный интервал
\onehalfspacing

\newlength\MyIndent
\setlength\MyIndent{1.25cm}
\setlength{\parindent}{\MyIndent} % Абзацный отступ
\frenchspacing            % Отключение лишних отступов после точек
\KOMAoptions{%
    DIV=calc,         % Пересчёт геометрии
    numbers=endperiod % точки после номеров разделов
}

                            % Консервативный вариант:
%\usepackage                % ручное задание геометрии
%[%                         % (не рекомендуется в проф. типографии)
%  margin = 2.5cm,
  %includefoot,
  %footskip = 1cm
%] %
%  {geometry}

%%% Заголовки

\ifthenelse{\equal{\theworktype}{2}}{%
\KOMAoptions{%
    numbers=endperiod,% точки после номеров разделов
    headings=normal,   % размеры заголовков поменьше стандартных
    chapterprefix=true,% Печатать слово Глава в магистерской
    appendixprefix=true% Печатать слово Приложение
}
}

% шрифт для оформления глав и названия содержания
\newcommand{\SuperFont}{\Large\sffamily\bfseries}

% Заголовок главы
\ifthenelse{\value{worktype} > 1}{%
\renewcommand{\SuperFont}{\Large\normalfont\sffamily}
\newcommand{\CentSuperFont}{\centering\SuperFont}
\usepackage{fncychap}
\ChNameVar{\SuperFont}
\ChNumVar{\CentSuperFont}
\ChTitleVar{\CentSuperFont}
\ChNameUpperCase
\ChTitleUpperCase
}

% Заголовок (под)раздела с абзацного отступа
\addtokomafont{sectioning}{\hspace{\MyIndent}}

\renewcommand*{\captionformat}{~---~}
\renewcommand*{\figureformat}{Listing~\thefigure}

% Плавающие листинги
\usepackage{float}
\floatstyle{ruled}
\floatname{ListingEnv}{Листинг}
\newfloat{ListingEnv}{htbp}{lol}[section]

% точка после номера листинга
\makeatletter
\renewcommand\floatc@ruled[2]{{\@fs@cfont #1.} #2\par}
\makeatother


%%% Оглавление
\usepackage{tocloft}

% шрифт и положение заголовка
\ifthenelse{\value{worktype} > 1}{%
\renewcommand{\cfttoctitlefont}{\hfil\SuperFont\MakeUppercase}
}{
\renewcommand{\cfttoctitlefont}{\hfil\SuperFont}
}

% слово Глава
\usepackage{calc}
\ifthenelse{\value{worktype} > 1}{%
\renewcommand{\cftchappresnum}{Chapter }
\addtolength{\cftchapnumwidth}{\widthof{Chapter }}
}

\newcommand{\setupname}[1]{%
  \addtocontents{toc}{%
    \unexpanded{\unexpanded{%
      \renewcommand{\cftchappresnum}{#1 }%
      \setlength\cftchapnumwidth{\widthof{\bfseries #1 }}%
      \addtolength\cftchapnumwidth{\fixedchapnumwidth}%
    }}%
  }%
}
\AtBeginDocument{\edef\fixedchapnumwidth{\the\cftchapnumwidth}}

% Очищаем оформление названий старших элементов в оглавлении
\ifthenelse{\value{worktype} > 1}{%
\renewcommand{\cftchapfont}{}
\renewcommand{\cftchappagefont}{}
}{
\renewcommand{\cftsecfont}{}
\renewcommand{\cftsecpagefont}{}
}

% Точки после верхних элементов оглавления
\renewcommand{\cftsecdotsep}{\cftdotsep}
%\newcommand{\cftchapdotsep}{\cftdotsep}

\ifthenelse{\value{worktype} > 1}{%
    \renewcommand{\cftchapaftersnum}{.}
}{}
\renewcommand{\cftsecaftersnum}{.}
\renewcommand{\cftsubsecaftersnum}{.}
\renewcommand{\cftsubsubsecaftersnum}{.}

%%% Списки (enumitem)

\usepackage {enumitem}      % Списки с настройкой отступов
\setlist %
{ %
  leftmargin = \parindent, itemsep=.5ex, topsep=.4ex
} %

% По ГОСТу нумерация должны быть буквами: а, б...
%\makeatletter
%    \AddEnumerateCounter{\asbuk}{\@asbuk}{м)}
%\makeatother
%\renewcommand{\labelenumi}{\asbuk{enumi})}
%\renewcommand{\labelenumii}{\arabic{enumii})}

%%% Таблицы: выбрать более подходящие

\usepackage{booktabs} % считаются наиболее профессионально выполненными
%\usepackage{ltablex}
%\newcolumntype {L} {>{---}l}

%%% Библиография

\usepackage{csquotes}        % Оформление списка литературы
\usepackage[
  backend=biber,
  hyperref=auto,
  sorting=none, % сортировка в порядке встречаемости ссылок
  language=auto,
  citestyle=gost-numeric,
  bibstyle=gost-numeric
]{biblatex}
\addbibresource{biblio.bib} % Файл с лит.источниками

% Настройка величины отступа в списке
\ifthenelse{\value{worktype} < 2}{%
\defbibenvironment{bibliography}
  {\list
     {\printtext[labelnumberwidth]{%
    \printfield{prefixnumber}%
    \printfield{labelnumber}}}
     {\setlength{\labelwidth}{\labelnumberwidth}%
      \setlength{\leftmargin}{\labelwidth}%
      \setlength{\labelsep}{\dimexpr\MyIndent-\labelwidth\relax}% <----- default is \biblabelsep
      \addtolength{\leftmargin}{\labelsep}%
      \setlength{\itemsep}{\bibitemsep}%
      \setlength{\parsep}{\bibparsep}}%
      \renewcommand*{\makelabel}[1]{\hss##1}}
  {\endlist}
  {\item}
}{}

% ----------------------------------------------------------------
% Настройка переносов и разрывов страниц

\binoppenalty = 10000      % Запрет переносов строк в формулах
\relpenalty = 10000        %

\sloppy                    % Не выходить за границы бокса
%\tolerance = 400          % или более точно
\clubpenalty = 10000       % Запрет разрывов страниц после первой
\widowpenalty = 10000      % и перед предпоследней строкой абзаца

% ----------------------------


% Стили для окружений типа Определение, Теорема...
% Оформление теорем (ntheorem)

\usepackage [thmmarks, amsmath] {ntheorem}
\theorempreskipamount 0.6cm

\theoremstyle {plain} %
\theoremheaderfont {\normalfont \bfseries} %
\theorembodyfont {\slshape} %
\theoremsymbol {\ensuremath {_\Box}} %
\theoremseparator {:} %
\newtheorem {mystatement} {Утверждение} [section] %
\newtheorem {mylemma} {Лемма} [section] %
\newtheorem {mycorollary} {Следствие} [section] %

\theoremstyle {nonumberplain} %
\theoremseparator {.} %
\theoremsymbol {\ensuremath {_\diamondsuit}} %
\newtheorem {mydefinition} {Определение} %

\theoremstyle {plain} %
\theoremheaderfont {\normalfont \bfseries} 
\theorembodyfont {\normalfont} 
%\theoremsymbol {\ensuremath {_\Box}} %
\theoremseparator {.} %
\newtheorem {mytask} {Задача} [section]%
\renewcommand{\themytask}{\arabic{mytask}}

\theoremheaderfont {\scshape} %
\theorembodyfont {\upshape} %
\theoremstyle {nonumberplain} %
\theoremseparator {} %
\theoremsymbol {\rule {1ex} {1ex}} %
\newtheorem {myproof} {Доказательство} %

\theorembodyfont {\upshape} %
%\theoremindent 0.5cm
\theoremstyle {nonumberbreak} \theoremseparator {\\} %
\theoremsymbol {\ensuremath {\ast}} %
\newtheorem {myexample} {Пример} %
\newtheorem {myexamples} {Примеры} %

\theoremheaderfont {\itshape} %
\theorembodyfont {\upshape} %
\theoremstyle {nonumberplain} %
\theoremseparator {:} %
\theoremsymbol {\ensuremath {_\triangle}} %
\newtheorem {myremark} {Замечание} %
\theoremstyle {nonumberbreak} %
\newtheorem {myremarks} {Замечания} %


% Титульный лист
% Макросы настройки титульной страницы
% В этот файл не предполагается вносить изменения

%\usepackage {showframe}

% Вертикальные отступы на титульной странице
\newcommand{\vgap}{\vspace{16pt}}

% Помещение города и даты в нижний колонтитул
\usepackage{scrlayer}
\DeclareNewLayer[
  foot,
  foreground,
  contents={%
    \raisebox{\dp\strutbox}[\layerheight][0pt]{%
      \parbox[b]{\layerwidth}{\centering Ростов-на-Дону\\ \thedate%
       \\\mbox{}
       }}%
  }
]{titlepage.foot.fg}
\DeclareNewPageStyleByLayers{titlepage}{titlepage.foot.fg}


\AtBeginDocument %
{ %
  %
  \begin{titlepage}
  %
    \thispagestyle{titlepage}

    {\centering
    %
    \MakeTextUppercase {МИНИСТЕРСТВО ОБРАЗОВАНИЯ И НАУКИ РФ}

    \vgap

    Федеральное государственное автономное образовательное\\
    учреждение высшего образования\\
    \MakeTextUppercase {Южный федеральный университет}

    \vgap

	Институт математики, механики и компьютерных наук
    имени~И.\,И.\,Воровича

    \vgap

    \direction

    \ifthenelse{\boolean{needchair}}{
    \vgap

    \thechair}{}

    \vspace* {\fill}

    \ifthenelse{\value{worktype} = 2}{%
    \me

    \vgap}{}

    {\usefont{T2A}{PTSansCaption-TLF}{m}{n}
    \MakeTextUppercase{\thetitle}}

    \ifthenelse{\value{worktype} = 2}{%
     \vgap

    Магистерская диссертация}{}
    \ifthenelse{\value{worktype} = 0}{
     \vgap

    Курсовая работа
    }{}%
    \ifthenelse{\value{worktype} = 1 \OR \value{worktype} = 3}{
     \vgap

    Выпускная квалификационная работа\\
    на степень бакалавра
    }{}%

    \vspace {\fill}

    \begin{flushright}
    \ifthenelse{\value{worktype} = 0 \OR
                \value{worktype} = 1 \OR
                \value{worktype} = 3}{
      \bystudent \ifthenelse{\value{worktype} = 0}{3}{4}\ курса\\
      \byme
    }{}

    \vgap

    Научный руководитель:\\
    к.т.н., с.н.с., доц. каф. ИВЭ
    А.~Н.~Литвиненко\\
    \ifthenelse{\value{worktype} = 2}{%
    Рецензент:\\
    к.ф.-м.н., доц., доц. каф. АДМ С.~С.~Михалкович}{}
	\end{flushright}
    \ifthenelse{\value{worktype} = 0}{
    \vspace{\fill}
            \begin{flushleft}
              \begin{tabular}{cc}
                \underline{\hspace{4cm}}&\underline{\hspace{5cm}}\\
                {\small оценка (рейтинг)} & {\small  подпись руководителя}\\
              \end{tabular}
            \end{flushleft}
    }{}
  	\vspace {\fill}
  %Ростов-на-Дону

    %\thedate

  }\end{titlepage}
  %
  %
  \tableofcontents
  %
  \clearpage
} %



% Команды для использования в тексте работы


% макросы для начала введения и заключения
\newcommand{\Ackns}{\addchap{Acknowledgements}}

\newcommand{\Intro}{\addchap{Introduction}}

\newcommand{\Goal}{\addchap{Goal statement}}

\newcommand{\Conc}{\addchap{Conclusion}}

% Правильные значки для нестрогих неравенств и пустого множества
\renewcommand {\le} {\leqslant}
\renewcommand {\ge} {\geqslant}
\renewcommand {\emptyset} {\varnothing}

% N ажурное: натуральные числа
\newcommand {\N} {\ensuremath{\mathbb N}}

% значок С++ — используйте команду \cpp
\newcommand{\cpp}{%
C\nolinebreak\hspace{-.05em}%
\raisebox{.2ex}{+}\nolinebreak\hspace{-.10em}%
\raisebox{.2ex}{+}%
}

% значок С# — используйте команду \cs
\newcommand{\cs}{%
C\nolinebreak\hspace{-.05em}%
\raisebox{.2ex}{\#}%
}

% Неразрывный дефис, который допускает перенос внутри слов,
% типа жёлто-синий: нужно писать жёлто"/синий.
\makeatletter
  \defineshorthand[english]{"/}{\babelhyphen{nobreak}}
  \addto\extrasenglish{
    \languageshorthands{english}
    \useshorthands{"}
  }
\makeatother



\endinput

% Конец файла


\begin{document}

Software systems are required to be reliable. For example, biomedical implants must be able to operate autonomously within patients, adapting to short term and long term changes, with required lifetimes in the order of decades, and one-minute downtime failure of banking software may lead to significant expenses. Safety and solidity of software may be achieved through testing and post-design quality assurance, but there is always a possibility to leave some branches of an operations scenario untested, leading to potential disaster.

In contrast with post-design testing, that doesn’t provide full
correctness guarantees, formal methods provide a systematic approach for developing complex systems in a correct by construction manner. One reason of advancement of
formal methods in comparison to testing is a maximal possible elimination of human errors. For instance, techniques known as model checking perform exhaustive search
through entire space of possible states of the system and evaluating system status in
every single state, completely excluding any possibility of non-specified behaviour (
presuming the correctness of specification).

One class of formal methods is advanced type systems provided by modern programming languages. Powerful yet lightweight formal verification techniques, provided by these languages are based on famous Curry-Howard correspondence --- a direct relationship between computer programs and mathematical
proofs --- that was discovered by the American mathematician Haskell Curry and logician
William Alvin Howard in late 1960s. A perfectly written Philip Wadler’s paper,
called ``Propositions as Types''~\cite{Wadler:2015:PT:2847579.2699407}
contains a great set of historical notes alongside with points of view on
Curry-Howard correspondence from different fields of
mathematics and computer science. Curry-Howard correspondence is of great value
for software developers because it provides a possibility to formulate desired
properties of programs in terms of types and automatically acquire proofs of
the correctness of those programs through type checking.

Advances in type theory and theory of programming languages led to the development and spreading of effects systems --- a particular kind of type-guided verification providing a possibility to separate pure computations from ones flavoured with a computational side effect, life, for instance, file system IO. These technique
was first given an account by Moggi~\cite{Moggi:1991:NCM:116981.116984} who
used monads to provide a denotational semantics to lambda calculus with effects, and then Wadler~\cite{Wadler:1992:EFP:143165.143169} gave monads a practical instantiation in Haskell programming language.

Computational effects control techniques got plenty of attention both in academic and software engineering communities. As applications, such as, for example, parsers combinators~\cite{monParsing} were explored, a lot more requirements and demands
were introduced: combine effects in a modular way, provide finer-grained control
for effects. This led to development of both monadic~\cite{Liang:1995:MTM:199448.199528} and alternative approaches~\cite{Mcbride:2008:APE:1348940.1348941}
~\cite{DBLP:journals/jlp/BauerP15}.

This work aims to research and development of programming languages features supporting explicit and precise control of computational effects. It addresses the problem
of construction of parser combinators libraries using three approaches to
effectful programming:

\begin{itemize}
  \item Monad transformers~\cite{Liang:1995:MTM:199448.199528} in~\texttt{Haskell}.
  \item Algebraic effects and effects handlers in a form of extensible.
  effects~\cite{Kiselyov:2013:EEA:2578854.2503791} in~\texttt{Haskell}
  \item~\texttt{Frank}~\cite{DBLP:conf/popl/LindleyMM17} programming language featuring first-class support of algebraic
  effects and effects handlers.
\end{itemize}

Main results of this work were presented in \nth{4} international conference ``Tools and Methods of Program Analysis 2017''~\cite{tmpa} in Moscow and conference
``Programming languages and compilers 2017'' in Rostov-on-Don.
Tell about scopus and ВАК publications.

Tell about structure of this thesis.



\input{src/chapter-1-fp.tex}

\chapter{Computational effects}
~\label{cpt-effects}

  Some fairly general words on why computational effects are useful: pure and impure programs, resource-sensitivity... Separate effects, combining several effects into one, modularity issues.

  \section{Functors and applicative functors}

  \section{Monads and monad transformers}

  \section{Algebraic effects and effects handlers}

\chapter{Building parser combinators as effectful programmes}
\label{cpt-parsers}

A parser is a necessary part of a broad range of software systems: from web browsers
to compilers. Parsers may be automatically generated or hand-written. Like any
software, parsers can carry implementation errors. One of the possible
methods of development of robust and correct-by-design software is using a programming
language with a rich type system. Modern programming languages offer facilities of
lightweight program verification using strict static typing discipline.

Parsing could be thought as a computation that operates over an input sequence of
characters, carrying some state (i.e. current position in input) and have a
possibility to fail. These features are computational effects. Chapter~\ref{cpt-effects}
introduced the most popular approaches to the construction of effectful computations and
this one show how these methods could be used to build parser combinators.

One of the approaches to parser construction that benefits from eloquent type system
is monadic parser combinators~\cite{monParsing}. Is has been widely accepted by the
community and was used to implement industrial-grade \texttt{Haskell} libraries, such
as \texttt{Parsec}~\cite{parsec} and \texttt{Attoparsec}~\cite{attoparsec}. This
approach represents parser as a monad and produces parsers powerful enough to admit
context-sensitive grammars. Section~\ref{cpt-parsers:monadic} gives an account to this kind of parsers.

A parser could be represented not only by monadic computation but also by an
applicative functor~\cite{Mcbride:2008:APE:1348940.1348941}. An applicative interface
is less restrictive than monadic one, but even though, it is possible to capture
context-free grammars. These parsers are described in
section~\ref{cpt-parsers:applicative}.

It is also feasible to represent an interface of a parser as an algebraic effect and
the process of analysis as a handler for this effect. Unlike monads and applicative functors,
algebraic effects do not still have wide support in popular programming languages.
Section~\ref{cpt-parsers:alg-eff} shows how parsing could be expressed using this
approach and provides examples in an experimental programming language~\texttt{Frank}
that has built-in support for algebraic effects and handlers.

  \section{Monadic parsers}
  \label{cpt-parsers:monadic}

    \subsection{Monads in Haskell}
      Monads were firstly injected into programming languages context as a tool to
      assign a denotational semantics to computational effects~\cite{Moggi:1991:NCM:116981.116984}. Later, the have been adopted as a programming
      paradigm and introduced into functional programming languages~\cite{Wadler:1992:EFP:143165.143169}. Monads are sometimes refereed as a ``programmable semicolon'' --- a powerful way to construct sequences of computation with possible side-effect. Afterwards, even more notions from category theory were given first-class support in modern programming languages, providing programmers with highly abstract, powerfully expressive and mathematically structured ways to build software.

      In \texttt{Haskell} programming language, monads are types that have an instance of \texttt{Monad} type class and satisfy three laws. They are used
      to distinguish pure computations from ones having some kind of side effect:
      mutable state, exceptions, non-determinism, etc.

      \begin{figure}[h]
      \begin{lstlisting}
class Monad m where
  (>>=)  :: m a -> (a -> m b)   -> m b
  (>>)   :: m a ->  m b         -> m b
  return ::   a                 -> m a
  fail   :: String -> m a
      \end{lstlisting}
      \caption{\texttt{Monad} type class}
      \label{listing:monadClass}
      \end{figure}

      \begin{figure}[h]
      \begin{lstlisting}
return a >>= k                  =  k a
m        >>= return             =  m
m        >>= (\x -> k x >>= h)  =  (m >>= k) >>= h
      \end{lstlisting}
      \caption{Monad laws}
      \label{listing:monadLaws}
      \end{figure}

      The \lstinline{>>=} operation, also known as \emph{monadic bind}, represents
      a, mentioned earlier, ``programmable semicolon''. It takes a value in a monadic context as it first argument, an action that transforms that value as a second argument and returns a transformed value in the same monadic context.

      Monads have broad usage in functional programming. They are first-class citizens
      in purely-functional languages like~\texttt{Haskell} and a wide range of~\texttt{Haskell}-libraries have monadic interface. Mainstream programming languages also
      employ specific monads in a form of build-in language constructions,
      i.e.~\texttt{LINQ} in~\texttt{C\#} or optionals in~\texttt{Swift}.

      As it was previously said, monads are used to characterise types of computations with a particular side effect. But what if a computation may
      potentially produce two or more effects? Then, means to combine several
      computational effects are needed. Monadic approach provide notion of
      ~\emph{monad transformer}~\cite{Liang:1995:MTM:199448.199528} --- a type that
      may add properties of a given monad to any other. Monad transformers are widely
      used in~\texttt{Haskell} to build computations carrying multiple side effects.


    \subsection{Parser as a Monad}

      Consider a simple type to represent a parser.

      \begin{figure}[h]
      \begin{lstlisting}
type Parser a = String -> Maybe (a,String)
      \end{lstlisting}
      \caption{Basic parser type}
      \label{listing:maybeParser}
      \end{figure}

      In this representation, parser is a
      function, taking input stream and returning a list of possible valid
      variants of analysis in conjunction with corresponding input stream
      remains. Empty list of result stands for completely unsuccessful attempt of
      parsing, whereas multiple results mean ambiguity.

      Types similar to \texttt{Parser a} may be treated as effectful computation.
      To represent computations with effects a concept of
      \texttt{Monad} is used in~\texttt{Haskell} programming language. This particular
      type could be made an instance of \texttt{Monad} type class.
      Comprehensive information about properties of parsers like one presented
      above may be found in paper~\cite{monParsing}.

      To extend capabilities and improve convenience of syntactic analysers, set of
      effects of parser could be expanded: it is handy to run parsers in a configurable
      environment or introduce logging. In this section two approaches to combination
      of computational effects will be considered: monad transformers and extensible
      effects.

    \subsection{Factorising parser into monad transformers stack}

  \section{Applicative parsing}
  \label{cpt-parsers:applicative}

  \section{Parsers as algebraic effects}
  \label{cpt-parsers:alg-eff}

  Algebraic effects and effects handlers provide an alternative to monads and monad
  transformers way to express effectful computations. Building parser combinators in
  term of algebraic effects and the process of parsing as their handlers is a solid
  model problem to find out strengths and weaknesses of this approach.

  This section describes leads to prototype implementations of parser
  combinators libraries in experimental programming language
  Frank~\cite{DBLP:conf/popl/LindleyMM17} which has first-class support for
  algebraic effects and effects handlers. Parsers may be represented either by
  combination of multiple effects, for instance, mutable state and possible failure,
  or may be expressed as a monolith effect signature. In conclusion, a
  note on expressive power of Franks implementation of algebraic effects and
  handlers is made.

  \subsection{As a combination of effects}

    Section~\ref{cpt-effects:alg-effects} gives an account on algebraic effects and
    effects handlers and, in particular, on programming with these concepts in Frank
    programming language. This section employs Frank to build a prototype of parser
    combinators library.

    \subsubsection{Defining parser combinators}

    As is has been already said, simple parser could be expressed as a computation
    with two effects: state of input stream and a possibility of failure. Thus,
    handling parsing means handling a combination of those two effects, that is done
    by composing handlers for failure and state (see
    listing~\ref{listing:parserHandlerCombo}).

    \begin{figure}[h]
    \begin{lstlisting}
parse : {[Error, State (List Char)] X} -> (List Char) -> Maybe X
parse p str = catch (state str p!)
    \end{lstlisting}
    \caption{Handling combination of state and failure}
    \label{listing:parserHandlerCombo}
    \end{figure}

    First parser that serves as a most basic building block in construction of
    more advanced ones is the~\emph{unconditional consumer}.
    It must take the first item
    of the input stream and yield it as a result, updating the state of the input
    stream with it's remains. In case of exhausted input, parser must fail. That
    is exactly the behaviour described by~\lstinline{item} function of listing
    ~\ref{listing:parserItemCombo}.

    \begin{figure}[h]
    \begin{lstlisting}
item : [Error, State (List Char)] Char
item! = on get! { nil -> fail
                | (x :: xs) -> put xs; x}
    \end{lstlisting}
    \caption{Parser consuming single item}
    \label{listing:parserItemCombo}
    \end{figure}

    Of course, unconditional consumption of the input stream without any actions
    doesn't make much sense. Actually, we would prefer consuming some items to others. Thus,~\emph{conditional consumer} that checks if an item satisfies a
    given predicate prior to consuming and fails otherwise, must be
    implemented (~\ref{listing:parserSatCombo}).

    \begin{figure}[h]
    \begin{lstlisting}
sat : {Char -> [Error, State (List Char)] Bool} ->
      [Error, State (List Char)] Char
sat p = on item! {c -> if (p c) {c} {fail}}
    \end{lstlisting}
    \caption{Conditional consumer}
    \label{listing:parserSatCombo}
    \end{figure}

    Having these basic building blocks, we are already able to construct
    practical parsers. A useful application of~\texttt{sat} is
    the~\texttt{char} parser that accepts a given character from the input
    stream (listing~\ref{parserCharCombo}).

    \begin{figure}[h]
    \begin{lstlisting}
char : Char -> [Error, State (List Char)] Char
char c = sat {x -> eqChar x c}
    \end{lstlisting}
    \caption{Parser for a given character}
    \label{listing:parserCharCombo}
    \end{figure}

    Besides accepting specific characters,~\texttt{sat} parser could be used
    to implement other basic parsers. For instance, if predicate
    ~\texttt{isLetter} determining weather of not given character is Latin letter is defined, we could supply it to~\texttt{sat} and acquire a parser for letters (listing~\ref{listing:parserLetterCombo}). The same could be done for decimal
    (or other) digits.

    \begin{figure}[h]`
    \begin{lstlisting}
letter : [Error, State (List Char)]Char
letter! = sat isLetter

digit : [Error, State (List Char)]Char
digit! = sat isDigit
    \end{lstlisting}
    \caption{Parsers letters and digits}
    \label{listing:parserLetterCombo}
    \end{figure}

    Now, being able to parse singular characters, we can make ourselves a task
    to accept sequences. That could be useful, for example, to parse terminals of
    some grammar. Here the main power of Frank's effect support comes in handy.
    As far as a string is essentially a list of characters, we can use the
    standard~\texttt{map} function in presence of~\texttt{Error}
    and~\texttt{State (List Char)}~\emph{abilities}.

    \begin{figure}[h]
    \begin{lstlisting}
string : (List Char) ->
         [Error, State (List Char)] (List Char)
string str = map char str
    \end{lstlisting}
    \caption{Parser for a given string}
    \label{listing:parserStrCombo}
    \end{figure}

    But individual chars and known-in-advance strings are not that interesting. Therefore, turn comes to actually building actual combinators: alternative
    and repetition.

    Consider the case we have to parsers $p_1$ and $p_2$, and we would like to
    construct a parser that accepts all inputs that are recognisable by both
    $p_1$ and $p_2$. In our setting we could implement this by deterministic
    choice: we apply $p_1$ and yield the result if it succeeds, otherwise we
    apply $p_2$ and initialise an error if it has failed, returning its
    result in case of success.

    \begin{figure}[h]
    \begin{lstlisting}
choose : {[Error, State (List Char)] X} ->
         {[Error, State (List Char)] X} ->
          [Error, State (List Char)] X
choose p1 p2 =
  on (parse p1 get!) { (right _)  -> p1!
                     | (left  _)  ->  on (parse p2 get!)
                       { (right _)   -> p2!
                       | (left err)  -> throw err
                       }
                     }
    \end{lstlisting}
    \caption{Alternative combinator}
    \label{listing:parserChooseCombo}
    \end{figure}

    For the simplest instance for~\texttt{choose} usage, reconsider
    ~\texttt{letter} and~\texttt{digit} parsers. We could combine those with the
    alternative combinator to accept alphanumeric characters
    (listing~\ref{listing:parserAlphanumCombo}).

    \begin{figure}[h]
    \begin{lstlisting}
alphanum : [Error, State (List Char)]Char
alphanum! = choose digit letter
    \end{lstlisting}
    \caption{Parser for alphanumerics}
    \label{listing:parserAlphanumCombo}
    \end{figure}

    The motivation for repetition combinators is the need to apply an already defined
    parser multiple times with no certainty about the amount of required applications.
    Repetition combinators are useful for problems like parsing a sequence
    of statements of a programming language. In parser combinators approach,
    repetition combinators are usually defined as two mutually recursive functions:
    ~\texttt{many} accepts the result of zero or more applications of its
    argument-parser $p$ and~\texttt{some} succeeds if $p$ is applicable at least
    once.

    \begin{figure}[h]
    \begin{lstlisting}
many : {[Error, State (List Char)]X} ->
        [Error, State (List Char)](List X)
many p = choose {some p} {nil}

some : {[Error, State (List Char)] X} ->
        [Error, State (List Char)](List X)
some p = p! :: many p
    \end{lstlisting}
    \caption{Repetition combinators}
    \label{listing:parserManyCombo}
    \end{figure}

    As an example of repetition combinator usage, consider parser for words ---
    a sequence of letters (listing~\ref{listing:parserWordCombo}).

    \begin{figure}[h]
    \begin{lstlisting}
word : [Error, State (List Char)] (List Char)
word! = some letter
    \end{lstlisting}
    \caption{Parser for words}
    \label{listing:parserWordCombo}
    \end{figure}

    Besides plain sequences, it is very common to have sequences separated by some
    kind of marker, for instance in CSV files. Therefore, a repetition with separation
    combinator could be useful. It could be implemented on top of~\texttt{many}
    combinator, see listing~\ref{listing:parserSepbyCombo}. Again,~\texttt{sepby}
    accepts any, including empty, sequence, while~\texttt{sepby1} requires at least
    one complete period.

    \begin{figure}[h]
    \begin{lstlisting}
sepby : {[Exception ParseError, State String]X} ->
        {[Exception ParseError, State String]Y} ->
        [Exception ParseError, State String](List X)
sepby p sep = choose {sepby1 p sep} {[]}

sepby1 : {[Exception ParseError, State String]X} ->
         {[Exception ParseError, State String]Y} ->
         [Exception ParseError, State String](List X)
sepby1 p sep = p! :: (many {sep!; p!})
    \end{lstlisting}
    \caption{Repetition with separation combinators}
    \label{listing:parserSepbyCombo}
    \end{figure}

    The discussed combinators

    \subsubsection{Case-study: parsing simplified Markdown}

    As a usage example for the developed library, consider the
    parser of simplified Markdown-like language. We take into account only basic
    constructions to keep the example concise: the language is limited to headers,
    plain paragraphs of text and unordered lists. The abstract syntax tree of
    the language is captured in Frank by the algebraic data type
    (listing~\ref{listing:parserMdAstCombo}).

      \begin{figure}[h]
      \begin{lstlisting}
data Document = Document (List Block)

data Block = Blank
           | Header (Pair Int Line)
           | Paragraph (List Line)
           | UnorderedList (List Line)

data Line = Empty | NonEmpty (List String)
      \end{lstlisting}
      \caption{Simplified Markdown AST}
      \label{listing:parserMdAstCombo}
      \end{figure}

      Again, to keep this example simple and concise, we do not consider any in-line
      formatting like bold or italic cases. Therefore in-line elements are represented
      as plain strings of characters. The line of text is either an empty line or
      non-empty one, and we use~\texttt{choose} combinator to represent this
      alternative (listing~\ref{listing:parserLineCombo}). An empty line is a sequence of
      spaces terminated with a newline character --- exactly this is captured by
      ~\texttt{emptyLine} parser: we use~\texttt{many} combinator supplying parser
      for a whitespace, then the parser for a newline character and, finally, the
      data constructor~\texttt{Empty} of~\texttt{Line} data type. To build a parser for
      a non-empty line, we need to refine the previous parser to be able to accept the
      meaningful contents. We use repetition with separation combinator~\texttt{sepby}
      to parse a sequence of words separated by spaces and then return an accepted sequence
      wrapped up in the~\texttt{NonEmpty} data constructor. The interesting
      thing here is the~\texttt{let}-binding. We need to accept and than throw away
      the newline character that terminates the line, but at the same time we need to save
      the meaningful contents. Hence we parse it and bind to a name for later use in
      data constructor that gets returned. In Frank we work with
      effectful functions in the same way like we do with the pure ones; therefore
      syntactic sugar similar to~\texttt{do}-notation becomes redundant.

      \begin{figure}[h]
      \begin{lstlisting}
line : [Exception ParseError, State String] Line
line! = choose emptyLine nonEmptyLine

emptyLine : [Exception ParseError, State String] Line
emptyLine! = many {char ' '};
             char '\n';
             Empty

nonEmptyLine : [Exception ParseError, State String] Line
nonEmptyLine! = many {char ' '};
                let ws = sepby word {char ' '}
                in char '\n'; NonEmpty ws
      \end{lstlisting}
      \caption{Parsers for lines}
      \label{listing:parserLineCombo}
      \end{figure}

      The abstract syntax described in listing~\ref{listing:parserMdAstCombo}
      has a data type to represent a block of a Markdown document. To be able to
      implement a parser for blocks we must implement a parser for every possible
      block type: header, paragraph or unordered list.

      \begin{figure}[h]
      \begin{lstlisting}
block : [Exception ParseError, State String] Block
block! = choose header {choose paragraph unorderedList}
      \end{lstlisting}
      \caption{Parser for a block of Markdown document}
      \label{listing:parserExprAppCombo}
      \end{figure}

      A Markdown header is a sequence of hash (\#) symbols followed by a
      non-empty line of text. The number of hashes represents the level of
      importance of the header, that is, one hash stands for HTML
      tag~\texttt{<h1>}, two hashes for ~\texttt{<h2>}, etc. Therefore, the parser
      needs to count the hashes and supply the number as an argument to the~\texttt{Header}
      data constructor. Exactly that is done by~\texttt{header} parser and, thanks to Franks
      applicative syntax for effects, we could supply the calls to inner parsers directly
      as the arguments to the resulting data constructor.

      \begin{figure}[h]
      \begin{lstlisting}
header : [Exception ParseError, State String] Block
header! = Header (length (some {char '#'})) line!
      \end{lstlisting}
      \caption{Direct applicative style of defining parsers}
      \label{listing:parserExprAppCombo}
      \end{figure}

      Parsers for paragraphs and unordered lists are similar. We use repetition
      combinator~\texttt{some} to accept a non-empty sequence of lines for a paragraph and
      a non-empty sequence of lines prefixed with a star for unordered list.

      \begin{figure}[h]
      \begin{lstlisting}
paragraph : [Exception ParseError, State String] Block
paragraph! = Paragraph (some nonEmptyLine)

unorderedList : [Exception ParseError, State String] Block
unorderedList! = UnorderedList (some (char '*'; line))
      \end{lstlisting}
      \caption{Direct applicative style of defining parsers}
      \label{listing:parserExprAppCombo}
      \end{figure}

      Finally, we are ready to implement a parser for the complete document as a
      non-empty sequence of blocks.

      \begin{figure}[h]
      \begin{lstlisting}
document : [Exception ParseError, State String] Document
document! = Document (some block)
      \end{lstlisting}
      \caption{Parser for the document}
      \label{listing:parserMdDocCombo}
      \end{figure}

      To parse the document, we use the~\texttt{parse} function that was defined earlier
      in this section, supplying the parser and the input string. If everything is
      correct, the resulting AST will be produced.


      \begin{figure}[h]
      \begin{lstlisting}
> parse document "# ToDo list\nThings to do today\n* Drink coffee\n* Write thesis\n"

right (Document ([Header 1 (NonEmpty (["ToDo", "list"])),
                  Paragraph ([NonEmpty (["Things", "to", "do", "today"])]),
                  UnorderedList ([NonEmpty (["Drink", "coffee"]),
                                 NonEmpty (["Write", "thesis"])])]))
      \end{lstlisting}
      \caption{Using the parser}
      \label{listing:parserExprMainCombo}
      \end{figure}

      The implemented parser is a simple example of usage of parser combinators library
      implemented on top of algebraic effects and handlers abstraction. It gives a
      good example of usage of the facilities that Frank provides to build
      effectful computations.

    \subsubsection{Discussion}

    Frank provides convenient and expressive features for programming with
    algebraic effects and handlers. Frank's native support for computations with
    multiple effects doesn't require programmes to deal with much of boilerplate.
    When the collection of effects of the computation is determined, there is nothing
    more holding the programmer from doing his job: no need to wrap one's head around
    complex types like monad transformers stacks and static effects ordering.
    In addition, direct applicative style of defining effectful computations often
    leaves in unnecessary to give names to interim results, thus making code
    easier to follow.

    Because Frank is in its infancy, it lacks some useful language features that
    are habitual from more mature functional languages. For example, there is no
    way to declare any short-cut for part of type signature; hence the
    effect peg must be included in type signature of every parser combinator, making
    source code a bit more verbose than in could be.

    Representing parser as a computation combining several side effects is a
    triumph of modularity. Nevertheless, it somehow abuses one of the most important
    points about algebraic effects and effect handlers: the independence of
    syntax and semantics. Of course, parsing is perfectly representable by the
    combination of several algebraic effects, but it might also be interesting
    and useful to represent it as a separate effect signature to be able to assign
    different interpretations to the same parser's syntax. The potential usefulness
    of this is discussed in the next subsection.


  \subsection{As a standalone effect}




\Conc

This thesis contributes to the development of approaches to effectful computation.
Implementations of three parser combinators libraries based on three
effectful computation frameworks are presented: one based on Haskell
monad transformers~\cite{mdParse};
another implemented with extensible effects~\cite{extEffParsers}~
--- an embedding of algebraic effects
and effects handlers into Haskell; and a third one is a prototype implementation
of a parser combinators library in Frank~\cite{frankoparsec}~---
an experimental programming language
with native support for programming with algebraic effects and effects handlers.

The developed libraries demonstrate advantages and disadvantages of the considered
approaches. The thesis also gives an account to performance benchmarking of the monad
transformers and extensible effects based libraries.

While building the prototype parsing library in Frank, we found out the limit of
expressibility of algebraic effects in Frank: currently, it is impossible to
declare effects which would have commands yielding results of different types;
thus making impossible to implement full"/fledged parser as a monolithic effect.
A possible future work may include an extension of Frank's effect system to support
existential quantification or GART"/like syntax for commands of effect interfaces.

The last chapter of the thesis describes an application of functional programming
language with side"/effects control to development of real"/life software. We
present the architecture and explain the implementation details of a distributed
system for real"/time student activity monitoring
``Students Big Brother'''~\cite{sbbRepo}. We exploits the Haskell's type and effect
systems to make the implementation reliable and maintainable. The server"/side share
the domain types with the client code thus making it impossible fir them to diverge.
The server"/side effects is structured with monad transformers and the data collection
daemon code uses extensible effects. We report our experience on how advanced type system
features may improve the maintainability of a code base, make a development process
more structured, and, as a result, lead to a reliable software.



% Печать списка литературы (библиографии)
\printbibliography[%{}
    heading=bibintoc%
    ,title=Bibliography % если хочется это слово
]
% Файл со списком литературы: biblio.bib
% Подробно по оформлению библиографии:
% см. документацию к пакету biblatex-gost
% http://ctan.mirrorcatalogs.com/macros/latex/exptl/biblatex-contrib/biblatex-gost/doc/biblatex-gost.pdf
% и огромное количество примеров там же:
% http://mirror.macomnet.net/pub/CTAN/macros/latex/contrib/biblatex-contrib/biblatex-gost/doc/biblatex-gost-examples.pdf

\end{document}
