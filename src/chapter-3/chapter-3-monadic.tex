  \section{Monadic parsers}
  \label{cpt-parsers:monadic}

    \subsection{Monads in Haskell}

    \subsection{Parser as a Monad}

Consider a simple type to represent a parser.

\begin{lstlisting}
type Parser a = String -> [(a,String)]
\end{lstlisting}

In this representation, parser is a
function, taking input stream and returning a list of possible valid
variants of analysis in conjunction with corresponding input stream
remains. Empty list of result stands for completely unsuccessful attempt of
parsing, whereas multiple results mean ambiguity.

Types similar to \texttt{Parser a} may be treated as effectful computation. In this
particular example, effect of non-determinism is exploited to express ambiguity
of parsing. To represent computations with effects a concept of \texttt{Monad} is used in~\texttt{Haskell} programming language. Comprehensive information about
properties of parsers like one presented above may be found in
paper~\cite{monParsing}.

To extend capabilities and improve convenience of syntactic analysers, set of
effects of parser could be expanded: it is handy to run parsers in a configurable
environment or introduce logging. In this section two approaches to combination
of computational effects will be considered: monad transformers and extensible
effects.

    \subsection{Factorising parser into monad transformers stack}