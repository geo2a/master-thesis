\chapter{Building parser combinators as effectful programmes}
\label{cpt-parsers}

A parser is a necessary part of a broad range of software systems: from web browsers
to compilers. Parsers may be automatically generated or hand-written. Like any
software, parsers can carry implementation errors. One of the possible
methods of development of robust and correct-by-design software is using a programming
language with a rich type system. Modern programming languages offer facilities of
lightweight program verification using strict static typing discipline.

Parsing could be thought as a computation that operates over an input sequence of
characters, carrying some state (i.e. current position in input) and have a
possibility to fail. These features are computational effects. Chapter~\ref{cpt-effects}
introduced the most popular approaches to the construction of effectful computations and
this one show how these methods could be used to build parser combinators.

One of the approaches to parser construction that benefits from eloquent type system
is monadic parser combinators~\cite{monParsing}. Is has been widely accepted by the
community and was used to implement industrial-grade \texttt{Haskell} libraries, such
as \texttt{Parsec}~\cite{parsec} and \texttt{Attoparsec}~\cite{attoparsec}. This
approach represents parser as a monad and produces parsers powerful enough to admit
context-sensitive grammars. Section~\ref{cpt-parsers:monadic} gives an account to this kind of parsers.

A parser could be represented not only by monadic computation but also by an
applicative functor~\cite{Mcbride:2008:APE:1348940.1348941}. An applicative interface
is less restrictive than monadic one, but even though, it is possible to capture
context-free grammars. These parsers are described in
section~\ref{cpt-parsers:applicative}.

It is also feasible to represent an interface of a parser as an algebraic effect and
the process of analysis as a handler for this effect. Unlike monads and applicative functors,
algebraic effects do not still have wide support in popular programming languages.
Section~\ref{cpt-parsers:alg-eff} shows how parsing could be expressed using this
approach and provides examples in an experimental programming language~\texttt{Frank}
that has built-in support for algebraic effects and handlers.

  \section{Monadic parsers}
  \label{cpt-parsers:monadic}

    \subsection{Monads in Haskell}
      Monads were firstly injected into programming languages context as a tool to
      assign a denotational semantics to computational effects~\cite{Moggi:1991:NCM:116981.116984}. Later, the have been adopted as a programming
      paradigm and introduced into functional programming languages~\cite{Wadler:1992:EFP:143165.143169}. Monads are sometimes refereed as a ``programmable semicolon'' --- a powerful way to construct sequences of computation with possible side-effect. Afterwards, even more notions from category theory were given first-class support in modern programming languages, providing programmers with highly abstract, powerfully expressive and mathematically structured ways to build software.

      In \texttt{Haskell} programming language, monads are types that have an instance of \texttt{Monad} type class and satisfy three laws. They are used
      to distinguish pure computations from ones having some kind of side effect:
      mutable state, exceptions, non-determinism, etc.

      \begin{figure}[t]
      \begin{lstlisting}
class Monad m where
  (>>=)  :: m a -> (a -> m b)   -> m b
  (>>)   :: m a ->  m b         -> m b
  return ::   a                 -> m a
  fail   :: String -> m a
      \end{lstlisting}
      \caption{\texttt{Monad} type class}
      \label{listing:monadClass}
      \end{figure}

      \begin{figure}[t]
      \begin{lstlisting}
return a >>= k                  =  k a
m        >>= return             =  m
m        >>= (\x -> k x >>= h)  =  (m >>= k) >>= h
      \end{lstlisting}
      \caption{Monad laws}
      \label{listing:monadLaws}
      \end{figure}

      The \lstinline{>>=} operation, also known as \emph{monadic bind}, represents
      a, mentioned earlier, ``programmable semicolon''. It takes a value in a monadic context as it first argument, an action that transforms that value as a second argument and returns a transformed value in the same monadic context.

      Monads have broad usage in functional programming. They are first-class citizens
      in purely-functional languages like~\texttt{Haskell} and a wide range of~\texttt{Haskell}-libraries have monadic interface. Mainstream programming languages also
      employ specific monads in a form of build-in language constructions,
      i.e.~\texttt{LINQ} in~\texttt{C\#} or optionals in~\texttt{Swift}.

      As it was previously said, monads are used to characterise types of computations with a particular side effect. But what if a computation may
      potentially produce two or more effects? Then, means to combine several
      computational effects are needed. Monadic approach provide notion of
      ~\emph{monad transformer}~\cite{Liang:1995:MTM:199448.199528} --- a type that
      may add properties of a given monad to any other. Monad transformers are widely
      used in~\texttt{Haskell} to build computations carrying multiple side effects.


    \subsection{Parser as a Monad}

      Consider a simple type to represent a parser.

      \begin{figure}[t]
      \begin{lstlisting}
type Parser a = String -> Maybe (a,String)
      \end{lstlisting}
      \caption{Basic parser type}
      \label{listing:maybeParser}
      \end{figure}

      In this representation, parser is a
      function, taking input stream and returning a list of possible valid
      variants of analysis in conjunction with corresponding input stream
      remains. Empty list of result stands for completely unsuccessful attempt of
      parsing, whereas multiple results mean ambiguity.

      Types similar to \texttt{Parser a} may be treated as effectful computation.
      To represent computations with effects a concept of
      \texttt{Monad} is used in~\texttt{Haskell} programming language. This particular
      type could be made an instance of \texttt{Monad} type class.
      Comprehensive information about properties of parsers like one presented
      above may be found in paper~\cite{monParsing}.

      To extend capabilities and improve convenience of syntactic analysers, set of
      effects of parser could be expanded: it is handy to run parsers in a configurable
      environment or introduce logging. In this section two approaches to combination
      of computational effects will be considered: monad transformers and extensible
      effects.

    \subsection{Factorising parser into monad transformers stack}

  \section{Applicative parsing}
  \label{cpt-parsers:applicative}

  \section{Parsers as algebraic effects}
  \label{cpt-parsers:alg-eff}

  Algebraic effects and effects handlers provide an alternative to monads and monad
  transformers way to express effectful computations. Building parser combinators in
  term of algebraic effects and the process of parsing as their handlers is a solid
  model problem to find out strengths and weaknesses of this approach.

  This section describes leads to prototype implementations of parser
  combinators libraries in experimental programming language
  Frank~\cite{DBLP:conf/popl/LindleyMM17} which has first-class support for
  algebraic effects and effects handlers. Parsers may be represented either by
  combination of multiple effects, for instance, mutable state and possible failure,
  or may be expressed as a monolith effect signature. In conclusion, a
  note on expressive power of Franks implementation of algebraic effects and
  handlers is made.

  \subsection{As a combination of effects}

    Section~\ref{cpt-effects:alg-effects} gives an account on algebraic effects and
    effects handlers and, in particular, on programming with these concepts in Frank
    programming language. This section employs Frank to build a prototype of parser
    combinators library.

    As is has been already said, simple parser could be expressed as a computation
    with two effects: state of input stream and a possibility of failure. Thus,
    handling parsing means handling a combination of those two effects, that is done
    by composing handlers for failure and state (see
    listing~\ref{listing:parserHandlerCombo}).

    \begin{figure}[t]
    \begin{lstlisting}
parse : {[Error, State (List Char)] X} -> (List Char) -> Maybe X
parse p str = catch (state str p!)
    \end{lstlisting}
    \caption{Handling combination of state and failure}
    \label{listing:parserHandlerCombo}
    \end{figure}

    First parser that serves as a most basic building block in construction of
    more advanced ones is the~\emph{unconditional consumer}.
    It must take the first item
    of the input stream and yield it as a result, updating the state of the input
    stream with it's remains. In case of exhausted input, parser must fail. That
    is exactly the behaviour described by~\lstinline{item} function of listing
    ~\ref{listing:parserItemCombo}.

    \begin{figure}[t]
    \begin{lstlisting}
item : [Error, State (List Char)] Char
item! = on get! { nil -> fail
                | (x :: xs) -> put xs; x}
    \end{lstlisting}
    \caption{Parser consuming single item}
    \label{listing:parserItemCombo}
    \end{figure}

    Of course, unconditional consumption of the input stream without any actions
    doesn't make much sense. Actually, we would prefer consuming some items to others. Thus,~\emph{conditional consumer} that checks if an item satisfies a
    given predicate prior to consuming and fails otherwise, must be
    implemented (~\ref{listing:parserSatCombo}).

    \begin{figure}[t]
    \begin{lstlisting}
sat : {Char -> [Error, State (List Char)] Bool} ->
      [Error, State (List Char)] Char
sat p = on item! {c -> if (p c) {c} {fail}}
    \end{lstlisting}
    \caption{Conditional consumer}
    \label{listing:parserSatCombo}
    \end{figure}

  \subsection{As a standalone effect}


