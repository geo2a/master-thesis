\chapter{Computational effects}
~\label{cpt-effects}

  % Some fairly general words on why computational effects are useful: pure and impure programs, resource-sensitivity... Separate effects, combining several effects into one, modularity issues.

  Recent advances in theory of programming languages led to development and spreading
  of functional programming languages with advanced type systems. This provide
  software engineers a possibility to encode system specification in type-level, enforcing statically checked guaranties of correctness. A large cluster of errors is
  introduced into programs by uncontrolled side effects such as file system IO,
  network communication and mutable state. When many mainstream programming languages such as C++, Java and C\# follow static type discipline, they do not track
  side-effects, thus making it harder to reason about program correctness. In contrast,
  pure languages like Haskell and Idris forbid programs to execute effectful code
  without special type annotations.

  \begin{figure}[t]
  \begin {lstlisting}
  int plus(int x, int y) {
    print("I'm mutating the World without you noticing!");
    return x + y;
  }
  \end{lstlisting}
  \caption{Uncontrollable side-effect in impure language}
  \label{listing:effectfulPlus}
  \end{figure}

  \begin{figure}[t]
  \begin{lstlisting}
plus :: Int -> Int -> Int
plut x y = x + y
  \end{lstlisting}
  \caption{Pure Haskell function}
  \label{listing:purePlusHaskell}
  \end{figure}

  \begin{figure}[t]
  \begin{lstlisting}
plusIO :: Int -> Int -> IO Int
plutIO x y = do
  print "I'm mutating the World, but you know it"
  return (x + y)
  \end{lstlisting}
  \caption{Pure Haskell function}
  \label{listing:purePlusHaskell}
  \end{figure}

  After initial incorporation of side-effect control techniques in Haskell in
  form of Monads~\cite{Wadler:1992:EFP:143165.143169}, effects systems got a lot
  of development. Monadic approach has been enriched with the concept of monad
  transformers~\cite{Liang:1995:MTM:199448.199528} to provide a modular way of
  expressing computations with multiple side-effects. Even though monad
  transformers has been widely accepted as a modular approach to side effects control,
  they have some major drawbacks, that are addressed, for instance, by Kiselyov at al.~\cite{Kiselyov:2013:EEA:2578854.2503791}, and alternative approach --- Algebraic effects and effects handlers --- was proposed~\cite{DBLP:journals/jlp/BauerP15}
  ~\cite{Kiselyov:2013:EEA:2578854.2503791}. Some studies were carried out to
  compare expressive power of these approaches~\cite{DBLP:journals/corr/ForsterKLP16}.
  This chapter introduces main approaches to construction of effectful computation
  by example, with illustrations in Haskell and Frank programming languages.

  \section{Monads and monad transformers}

  \section{Algebraic effects and effects handlers}
  ~\label{cpt-effects:alg-effects}