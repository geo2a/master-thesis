\Conc

This thesis contributes to the development of approaches to effectful computation.
Implementations of three parser combinators libraries based on three
effectful computation frameworks are presented: one based on Haskell
monad transformers~\cite{mdParse};
another implemented with extensible effects~\cite{extEffParsers}~
--- an embedding of algebraic effects
and effects handlers into Haskell; and a third one is a prototype implementation
of a parser combinators library in Frank~\cite{frankoparsec}~---
an experimental programming language
with native support for programming with algebraic effects and effects handlers.

The developed libraries demonstrate advantages and disadvantages of the considered
approaches. The thesis also gives an account to performance benchmarking of the monad
transformers and extensible effects based libraries.

While building the prototype parsing library in Frank, we found out the limit of
expressibility of algebraic effects in Frank: currently, it is impossible to
declare effects which would have commands yielding results of different types;
thus making impossible to implement full"/fledged parser as a monolithic effect.
A possible future work may include an extension of Frank's effect system to support
existential quantification or GART"/like syntax for commands of effect interfaces.

The last chapter of the thesis describes an application of functional programming
language with side"/effects control to development of real"/life software. We
present the architecture and explain the implementation details of a distributed
system for real"/time student activity monitoring
``Students Big Brother'''~\cite{sbbRepo}. We exploits the Haskell's type and effect
systems to make the implementation reliable and maintainable. The server"/side share
the domain types with the client code thus making it impossible fir them to diverge.
The server"/side effects is structured with monad transformers and the data collection
daemon code uses extensible effects. We report our experience on how advanced type system
features may improve the maintainability of a code base, make a development process
more structured, and, as a result, lead to a reliable software.

