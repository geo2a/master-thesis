\Conc

This thesis contributes to the development of approaches to effectful computation.
Implementations of three parser combinators libraries based on three
effectful computation frameworks are presented: one based on Haskell monad transformers;
another implemented with extensible effects~--- an embedding of algebraic effects
and effects handlers into Haskell; and a third one is a prototype implementation
of a parser combinators library in Frank~--- an experimental programming language
with native support for programming with algebraic effects and effects handlers.

The developed libraries demonstrate advantages and disadvantages of the considered
approaches. The thesis also gives an account to performance benchmarking of the monad
transformers and extensible effects based libraries.

By building the prototype parsing library in Frank, we found out the limit of
expressibility of algebraic effects in Frank: currently, it is impossible to
declare effects which would have commands yielding results of different types;
thus making impossible to implement full"/fledged parser as a monolithic effect.
A possible future work may include an extension of Frank's effect system to support
existential quantification for commands of effect interfaces.

The last chapter of the thesis describes an application of functional programming
language with side"/effects control to development of real"/life software. We
present the architecture and explain the implementation details of a distributed
system. We convey out experience on how advanced type system features may improve
the maintainability of a code base, make a development process more structured,
and, as a result, lead to a more reliable software.

