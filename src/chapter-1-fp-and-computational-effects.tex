\chapter{Functional programming and computational effects}
~\label{cpt-effects}

Recent advances in theory of programming languages led to development and spreading
of functional programming languages with advanced type systems. This provide
software engineers a possibility to encode system specification in type-level, enforcing
statically checked guaranties of correctness. A large cluster of errors is
introduced into programs by uncontrolled side effects such as file system IO,
network communication and mutable state. When many mainstream programming languages such as \cpp,
Java and \cs~follow static type discipline, they do not track
side-effects, thus making it harder to reason about program correctness. In contrast,
pure languages like Haskell and Idris forbid programs to execute effectful code
without special type annotations.

\begin{figure}[h]
\begin {lstlisting}
int plus(int x, int y) {
  print("I'm mutating the World without you noticing!");
  return x + y;
}
\end{lstlisting}
\caption{Uncontrollable side-effect in an impure language}
\label{listing:effectfulPlus}
\end{figure}

\begin{figure}[h]
\begin{lstlisting}
plus :: Int -> Int -> Int
plut x y = x + y
\end{lstlisting}
\caption{Pure Haskell function}
\label{listing:purePlusHaskell}
\end{figure}

\begin{figure}[h]
\begin{lstlisting}
plusIO :: Int -> Int -> IO Int
plutIO x y = do
print "I'm mutating the World, but you know it"
return (x + y)
\end{lstlisting}
\caption{Pure Haskell function}
\label{listing:purePlusHaskell}
\end{figure}

After initial incorporation of side-effect control techniques in Haskell in
form of Monads~\cite{Wadler:1992:EFP:143165.143169}, effects systems got a lot
of development. Monadic approach has been enriched with the concept of monad
transformers~\cite{Liang:1995:MTM:199448.199528} to provide a modular way of
expressing computations with multiple side-effects. Even though monad
transformers has been widely accepted as a modular approach to side effects control,
they have some major drawbacks which are addressed, for instance, by Kiselyov at al.~\cite{Kiselyov:2013:EEA:2578854.2503791}, and alternative approach --- Algebraic effects and effects handlers --- was proposed~\cite{DBLP:journals/jlp/BauerP15}
~\cite{Kiselyov:2013:EEA:2578854.2503791}. Some studies were carried out to
compare expressive power of these approaches~\cite{DBLP:journals/corr/ForsterKLP16}.

% This chapter gives an overview of approaches to construction of effectful computation
% and intriduces them by example, with illustrations in Haskell and Frank programming languages.

% \section{Effectful Domain-specific languages}

% \section{Parser combinators}

% A parser is a necessary part of a broad range of software systems: from web browsers
% to compilers. Parsers may be automatically generated or hand-written. Like any
% software, parsers can carry implementation errors. One of the possible
% methods of development of robust and correct-by-design software is using a programming
% language with a rich type system. Modern programming languages offer facilities of
% lightweight program verification using strict static typing discipline.

% Parsing could be thought as a computation that operates over an input sequence of
% characters, carrying some state (i.e. current position in input) and have a
% possibility to fail. These features are computational effects. Chapter~\ref{cpt-effects}
% introduced the most popular approaches to the construction of effectful computations and
% this one show how these methods could be used to build parser combinators.

% One of the approaches to parser construction that benefits from eloquent type system
% is monadic parser combinators~\cite{monParsing}. Is has been widely accepted by the
% community and was used to implement industrial-grade \texttt{Haskell} libraries, such
% as \texttt{Parsec}~\cite{parsec} and \texttt{Attoparsec}~\cite{attoparsec}. This
% approach represents parser as a monad and produces parsers powerful enough to admit
% context-sensitive grammars. Section~\ref{cpt-parsers:monadic} gives an account to this kind of parsers.

% A parser could be represented not only by monadic computation but also by an
% applicative functor~\cite{Mcbride:2008:APE:1348940.1348941}. An applicative interface
% is less restrictive than monadic one, but even though, it is possible to capture
% context-free grammars. These parsers are described in
% section~\ref{cpt-parsers:applicative}.

% It is also feasible to represent an interface of a parser as an algebraic effect and
% the process of analysis as a handler for this effect. Unlike monads and applicative functors,
% algebraic effects do not still have wide support in popular programming languages.
% Section~\ref{cpt-parsers:alg-eff} shows how parsing could be expressed using this
% approach and provides examples in an experimental programming language~\texttt{Frank}
% that has built-in support for algebraic effects and handlers.