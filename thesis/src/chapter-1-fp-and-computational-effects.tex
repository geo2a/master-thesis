\chapter{A historical retrospective on computational effects}
~\label{cpt-effects}

Recent advances in the theory of programming languages led to the development and spreading
of functional programming languages with advanced type systems. This provide
software engineers with a possibility to encode system specification in type"/level, enforcing
statically checked guarantees of correctness. A large cluster of errors is
introduced into programs by uncontrolled side effects such as file system IO,
network communication and mutable state. While many mainstream programming languages
such as \cpp, Java and \cs~follow static type discipline, they do not track
side-effects, thus making it harder to reason about program correctness. Being an impure
language, Java has a feature of~\emph{checked exceptions} that might be considered a very simple
effect system specialised for exception handling. In contrast, pure languages like Haskell and
Idris oblige all kinds of effectful programs to have special type markers.

\begin{figure}[h]
\begin {lstlisting}
int plus(int x, int y) {
  print("I'm mutating the World without you noticing!");
  return x + y;
}
\end{lstlisting}
\caption{Uncontrollable side-effect in an impure language}
\label{listing:effectfulPlus}
\end{figure}

\begin{figure}[h]
\begin{lstlisting}
plus :: Int -> Int -> Int
plut x y = x + y
\end{lstlisting}
\caption{Pure Haskell function}
\label{listing:purePlusHaskell}
\end{figure}

\begin{figure}[h]
\begin{lstlisting}
plusIO :: Int -> Int -> IO Int
plutIO x y = do
  print "I'm mutating the World, but you know it"
  return (x + y)
\end{lstlisting}
\caption{Impure Haskell function with an effect annotation}
\label{listing:purePlusHaskell}
\end{figure}

The development of approaches to the structuring of computations with side effects
and implementations of these approaches in programming languages has an exciting
history. This chapter gives a historical excursion.

\section{Origins of side effects control}

From the very beginning, programming languages have been created to be the lingua franca
for humans and machines. And the compilers must have been the translators.
Therefore, the evolution of programming languages and compilers is
massively affected by the dichotomy between the readability for humans and efficiency
of machine code generation. This dichotomy was one of the primary motivations for
programming language research: a field of theoretical computer science which
studies syntax and semantics of programming languages.

Software, be it a number cruncher or a compiler, or
an accounting system, is not likely to be developed and maintained by the efforts
of one person. Therefore, the syntax of a programming language must be readable,
and the semantics must be clear and adequate. Moreover, in some fields errors in
software are impossible to tolerate. The need to build correct programmes leads to the
development of pre and post-development software verification techniques. The first
category includes various kinds of testing: unit testing, test-driven development,
functional and integration testing. But the bad thing about testing is that it's,
in fact, impossible to~\emph{prove} that the system is correct.
Therefore, programming languages must have been enriched with a fundamental way
to construct only correct programmes, to make the incorrect ones unrepresentable.
This leads to the second category of software verification: the one based
on~\emph{formal methods} and, more specifically, type systems of programming
languages. It would be great to design the language which would reject incorrect
programmes. And that is exactly what an adequate type system can do.

So, every programming language has a set of built-in constructions and syntactic rules.
Programmers could take the constructions and combine them with the rules. But the programming
language also have a semantics~--- a way to assign a~\emph{meaning} to every ~\emph{term}
of the language. There are several approaches to describe programming language
semantics, including ~\emph{operational},~\emph{denotational}, and~\emph{axiomatic}.
And the task for the programming language designers is to construct
the language in a way when the terms with inadequate semantics would be rejected
by the type system.

A classical example of how type system could help to introduce correctness control into
a programming language is the difference between the untyped $\lambda$"/calculus and the simply
typed $\lambda$"/calculus. If we consider a property of~\emph{strong normalisation}
of $\beta$"/reduction as a criterion, then it is shown that for the untyped $\lambda$"/calculus
$\beta$"/reduction as a rewrite rule is not strongly normalising~--- i.e. there are
such things as non"/terminating computations or ``infinite loops''. Whereas
for the simply typed $\lambda$"/calculus $\beta$"/reduction is strongly normalising;
hence evaluation of any term would terminate in a finite number of steps. That is so
because constructions which lead to non"/termination, such as \textbf{Y}"/combinator,
are not~\emph{well"/typed} in the simply typed $\lambda$"/calculus.

Through the past years, a wide and deep research on typed lambda calculi was carried out,
and then it found applications in the theory of programming languages and in
functional programming. Both untyped (or~\emph{uni-typed}) and typed functional programming
languages have a strong theoretical background and have always been a platform for
researchers to explore applications of new ideas. In the late 1980s, typed functional programming
started to receive a lot of attention and plenty of ideas got implemented in experimental
languages. One of the primary and most promising tracks was the introduction of purely
functional programming~--- a style of building the structure and elements of computer
programmes, that treats all computation as the evaluation of mathematical functions. This
approach extensively uses notions of~\emph{immutability} and~\emph{referential transparency}.

And again, the development of programming languages was mostly driven by a dichotomy, but
this time between abstract beauty of purely functional programmes and necessity of impure ones.
That is, to be valuable, programmes must have~\emph{side-effects}: output their result to a file,
communicate with a user, mutate state, etc. But naive introduction of side-effects into
purely functional programmes was reported~\cite{Cartwright1994} to make
programmes denotational semantics unstable and non-extensible. Therefore, high demand for
solid mathematical foundations for controlling side"/effects emerged.

\section{Emergence of effect systems}

% One of motivations for development of effect systems was a problem of extensibility
% and modularity of programmes in pure functional programming languages.

Initially, purely"/functional programming languages with immutable data structures
were required to explicitly track all changes to the data in operation.
This was making programmes cumbersome and wordy. But then it was noted by
Moggi~\cite{Moggi:1991:NCM:116981.116984} that most programmes dealing with
state and other side"/effects follow the same pattern. He proposed to
use~\emph{monads}~--- a notion from category theory~--- to structure side-effecting
computations and factor the semantics via a new, generic, computational metalanguage that
saves repeated work, modularises, explains, cleans up reasoning by moving side"/conditions
into the type system.

\begin{itemize}
  \item Separate values $A$ from computations $T A$, which may have observable
  behaviour other than producing a value of type $A$.
  \item Given a category $C$, $T$ is an~\emph{endofunctor} $T:C \rightarrow C$, so can
  lift $f:A \rightarrow B$ to $T f:T A \rightarrow T B$, and this preserves
  identity and composition.
  \item There’s a~\emph{natural transformation} with components
   $\eta_A : A \rightarrow T A$ which expresses how values may be
  (uniformly) viewed as trivial computations.
  \item There’s a natural transformation $\mu_A : T (T A) \rightarrow T A$ that lets
  (uniformly) combine effectful behaviours, so computation of
  a computation could be treated as a computation
  \item Satisfying monadic laws.
\end{itemize}

Later, Wadler grasped~\cite{Wadler:1992:EFP:143165.143169} the Moggi's idea and
implemented it in the purely"/functional language Haskell. That was the first
well"/known incorporations of monads into a programming language.

    \begin{figure}[h]
    \begin{lstlisting}
class Monad m where
  (>>=)  :: m a -> (a -> m b)   -> m b
  (>>)   :: m a ->  m b         -> m b
  return ::   a                 -> m a
  fail   :: String -> m a
    \end{lstlisting}
    \caption{\texttt{Monad} type class in Haskell}
    \label{listing:monadClass}
    \end{figure}

    \begin{figure}[h]
    \begin{lstlisting}
return a >>= k                  =  k a
m        >>= return             =  m
m        >>= (\x -> k x >>= h)  =  (m >>= k) >>= h
    \end{lstlisting}
    \caption{Monad laws}
    \label{listing:monadLaws}
    \end{figure}

Monads were widely adopted by functional programming community and were used to
refactor and unify such techniques as list comprehensions, purely"/functional
exceptions and state"/threading. Many more monad instances were later discovered,
including functional parsers~\cite{monParsing}, ways to control concurrency,
logging and a lot of others. Moreover, it was discovered that monads are a convenient
way to build domain"/specific languages (DSLs).

Essentially, every monad represents one computational effect: state,
non"/determinism, exceptions, etc. But complex programmes is likely to require an
ability to produce multiple side effects. Therefore, a way to describe
computations with several computational effects in a modular manner were demanded.

The first attempt to formally approach the effects combination problem was made
by Wadler and King~\cite{DBLP:conf/fp/KingW92} who investigated a possibility to
combine monads with each other to yield a~\emph{combined monad}~--- a one possessing
properties of two or more monads. Then, Liang at al.~\cite{Liang:1995:MTM:199448.199528}
came up with the notion of~\emph{monad transformers}~--- building blocks for construction
of modular programming language interpreters. The major achievement of the paper
is a way to specify effects interactions through~\emph{lifting} and changing
semantics by reordering monads in~\emph{monad stacks}. Monad transformers were a
magnificent breakthrough. They strongly fortified in the Haskell community and were
considered a state"/of"/the"/art technique for a long time.

However, alternative approaches to the problem of the unstable denotational semantics
of pure functional languages with effects were investigated. Cartwright and Felleisen proposed
a formalism named~\emph{extended direct semantics}~\cite{Cartwright1994} almost at the same
time with Moggi's monads. That formalism represents effects as a generalisation of exceptions:
a computation may ``rise'' an effect, e.g. perform IO or mutate some kind of state, that
must be later handled by a centralised~\emph{authority} (similar to exception handler).
The idea of treating effects as exceptions was later studied more formally and the
notions of~\emph{algebraic effects}~\cite{DBLP:journals/jlp/BauerP15} by Bauer and Pretnar
and~\emph{effects handlers}~\cite{Plotkin2009} by Plotkin and Pretnar were introduced. These
abstractions received much attention lately and several implementations were introduced.

Currently, the problem of construction of effectful computations have several
main competing approaches: it is yet to analyse their similarities and differences,
to determine which one is better, or to find out if they are somehow the same.

\section{Current state of the field}

This section gives a short description of current state of computational effects:
primary research directions and support of computational effects in programming
languages.

The monadic approach is widely accepted in the functional programming community,
especially in Haskell. Haskell even have a special syntactic sugar,~\emph{do"/notation}, to
make monadic code look more ``imperative''. This notation then gets de-sugared
into a sequence of monadic~\emph{binds} and $\lambda$"/abstractions. For computations
with multiple side"/effects, monad transformers are considered the standard technique:
a lot of major Haskell libraries and applications are implemented on top of a
monad"/transformers stacks, for instance, monadic parser combinators library
Parsec~\cite{parsec}.

Despite the fact that monads are that popular in Haskell, there are implementations
of alternative approaches. Kiselyov at al. presented~\emph{Extensible Effects}~---
an approach alternative to monad transformers for construction computations with
several effects. Extensible effects are a continuation of Cartwright and Felleisen
work~\cite{Cartwright1994} but based on the mathematical concept of a~\emph{free monad}~--- a
method to derive a monad from an arbitrary functor. Extensible effects also may be
considered a Haskell embedding of algebraic effects and effects handlers. This thesis
explores applications of extensible effects to the construction of parser combinators
libraries, see chapter~\ref{cpt-alg-eff:ext-effects} for details.

Brady's Idris~\cite{DBLP:conf/lfmtp/Brady14} programming language has an effect
system based on algebraic effects and effects handers. Idris is a strict
dependently"/typed purely"/functional programming language, and its effect system
heavily employs dependent types.

Bauer and Pretnar created and effect system for their ML"/like language
Eff~\cite{DBLP:journals/corr/BauerP13}. The Eff itself is build around the concepts
of algebraic effects and effects handlers.

One of the most recent implementations of algebraic effects and effects handlers
is Frank programming language by Lindley at. al~\cite{DBLP:conf/popl/LindleyMM17}.
Frank is an experimental language
with native support for algebraic effects, it has special syntactic constructions
to define algebraic effects and their handlers, and is designed to make programming
with algebraic effects natural. Frank is discussed in more detail in chapter
~\ref{cpt-alg-eff:frank} of this thesis, and its usage to implement parser
combinators is explored.

Some studies were delivered to compare the expressive power of these
approaches~\cite{DBLP:journals/corr/ForsterKLP16}, which carried out the verdict
that monad transformers may be converted into introduced~\emph{modular algebraic
effects}: every~\emph{modular algebraic effect} gives rise to a monad transformer
and vice"/versa. However, the relative convenience of programming with these abstractions
is yet to be explored.

In conclusion, it could be said that nowadays both monadic approach and algebraic
effects with their handlers are popular to construct computations with multiple
side"/effects. Both abstractions have a wide range of implementations in programming
languages and the problem of unification and finding an ultimate approach to rule
all the effects has not yet been solved.
